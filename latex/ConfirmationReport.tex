\documentclass[oneside,12pt]{article} 

\usepackage[margin=2.5cm]{geometry} 
\usepackage{sidecap}
\usepackage{fullpage}
\geometry{a4paper}
\usepackage[final]{graphicx}
\usepackage{url}
\usepackage{amsmath, amssymb}
\usepackage{amsfonts}
\usepackage[mathscr]{euscript}
\usepackage{mathtools}
\usepackage{bbold}
\usepackage{float}
\usepackage{wrapfig}
\usepackage{sidecap}
\usepackage{caption}
\usepackage{subcaption}
\usepackage{multirow}
\usepackage{sidecap}
\usepackage{fancyhdr}
\usepackage{verbatim}
\usepackage[rflt]{floatflt}
\usepackage{titlesec}
\usepackage{gensymb}
\usepackage{enumerate}
\usepackage{cleveref} % allows referencing multiple figures in one go
\usepackage{pdflscape} % to allow table to be displayed in landscape mode
\usepackage{afterpage} % to allow table to be given entire page to itself
\usepackage{accents}
\usepackage{tikz} % drawing on images
\usepackage[sorting=none]{biblatex}
% \usepackage[backend=biber, bibencoding=utf8, style=authoryear, citestyle=authoryear]{biblatex}
\bibliography{Bibliography}
% \usepackage[square, comma, sort&compress]{natbib}
% \usepackage{bibentry}
\usepackage{lineno}

\linenumbers

\titleformat{\section}
  {\normalfont\fontsize{12}{15}\bfseries}{\thesection}{1em}{}

% \graphicspath{{/Users/alexandrarollings/Analysis/Confirmation/latex/figures/}}

% \renewcommand{\refname}{References}
\newcommand{\overbar}[1]{\mkern 1.5mu\overline{\mkern-1.5mu#1\mkern-1.5mu}\mkern
1.5mu}
\newcommand\brabar{\scalebox{.3}{(}\raisebox{-1.7pt}{$-$}\scalebox{.3}{)}}
\newcommand\brobor{\smash[b]{\raisebox{0.6\height}{\scalebox{0.5}{\tiny(}}{\mkern-1.5mu\scriptstyle-\mkern-1.5mu}\raisebox{0.6\height}{\scalebox{0.5}{\tiny)}}}}

\newcommand{\HRule}[1]{\rule{\linewidth}{#1}}     % Horizontal rule
% Remove 'In:' in Bibliography
\renewbibmacro{in:}{}
% Remove 'pp' in Bibliography
\DeclareFieldFormat{pages}{#1}


\makeatletter                            % Title
\def\printtitle{%                       
  {\centering \@title\par}}
\makeatother                                    

\makeatletter                            % Author
\def\printauthor{%                    
  {\centering \large \@author}}                
\makeatother                

\title{ \LARGE \textbf{Study of $B^{\pm} \rightarrow D^{*}h^{\pm}$ decays in the
interest of obtaining a direct measurement of $\phi_3$} \\ }

\author{
		Alexandra Rollings\\	
		Supervisor: Dr M. John\\	
}

\begin{document}
\begin{titlepage}
\begin{nolinenumbers}
\thispagestyle{empty} 

\begin{minipage}[c]{.15\linewidth}
  \includegraphics[width=\linewidth]{figures/Oxford}
\end{minipage}\hfill
\begin{minipage}[c]{.75\linewidth}
  \begin{flushright}
  \normalsize {Confirmation Report} 	% Subtitle
  \\ \normalsize \today			% Todays date
  \end{flushright}
\end{minipage}



\HRule{0.5pt} \\						% Upper rule
[2.0cm]
\printtitle 
\vspace{75pt}
\printauthor
\vfill
\begin{abstract}
\noindent
\\
Decays of the type $B^{\pm} \rightarrow D^{*}h^{\pm}$, where $D^{*} \rightarrow
[K^{\pm}\pi^{\pm}]_D$ $\pi^{0}\text{/}\gamma$ and $h=\pi/K$, are studied using
6.6 fb$^{-1}$ integrated luminosity in $pp$ collisions taken at the LHCb
experiment. This dataset consists of 3 fb${^{-1}}$ taken at center-of-mass
energies of 7 TeV, 8 TeV, and 3.6 fb${^{-1}}$ at 13 TeV. The techniques for
discriminating against combinatorial backgrounds are detailed and the physics
backgrounds are identified. A novel method to extract yields using a 2D maximum
likelihood fit is described and using a preliminary optimisation, the number of
events are reported. An extrapolation of event yields to the modes sensitive to
$\phi_3$ is then performed and initial measurements of physics observables are
presented.
\end{abstract}
\vfill
\end{nolinenumbers}
\end{titlepage}
\setcounter{page}{1}
\section{\normalsize Introduction}
% The Standard Model (SM) of particle physics is a quantum field theory which
% describes the electromagnetic, strong and weak interactions, and classifies
% all known elementary particles. Work over the last century has developed this
% successful theory, and many aspects of it have been validated by particle
% physics experiments around the world, a substantial contribution of which has
% been made by the CERN experiment in Geneva. However, the SM does fall short in
% a number of aspects, one of which being its failure to describe the
% matter-antimatter asymmetry we see in the observable Universe.  Quantities of
% the observable Universe is almost entirely matter
% dominated~\cite{UniAsymmetry}. It is therefore postulated that baryogenesis,
% the process that produced baryon asymmetry, occurred in the early Universe.
% Three Sakharaov conditions are required for this to be possible: I) baryon
% number violation, II) C and $CP$ violation, III) deviation from thermal
% equilibrium~\cite{Sakharov}.
The Standard Model (SM) of particle physics, the quantum field theory describing
three of the four fundamental forces, allows for $CP$ violating processes, but the
level at which these processes occur would need to be several orders of
magnitude larger to account for the Universal matter-antimatter asymmetry. This
suggest that new physics, beyond the SM, is needed to fully describe this
asymmetry. The LHCb experiment at CERN has been designed to collect
unprecedented samples of decays of $B$ mesons and study many $CP$-violating
processes of the SM. This report will describe an ongoing measurement of a SM
parameter critical to this endeavour.

This document gives an overview of the theoretical background and derivation of
physics observables in Sections~\ref{sec:theory} and~\ref{sec:observables}.
Section~\ref{sec:detector} contains a brief descripton of the LHCb detector and
neutral particle reconstruction. The selection of $B^{\pm}\rightarrow
(D^{*0}\rightarrow [K^{\pm}\pi^{\mp}]_D\pi^0/\gamma)h^{\pm}$ events is described
in Section~\ref{sec:selections}. Sections~\ref{sec:massfit}
and~\ref{sec:results} give a description of the 2D fit to the invariant mass
spectrums that will be used to extract the desired observables, and discuss the
results and current status of the fit, respectively. Lastly, a thesis outline is
then presented in Section~\ref{sec:thesis}. 

\section{\normalsize Theoretical Background} \label{sec:theory}
In the SM, the charged-current, weak interactions of quarks are described by the
SM Lagrangian term given in Eq.~\eqref{eq:CCLagrangian}.
\begin{equation}
  L_{CC}=-\frac{g}{\sqrt{2}}\overbar{q}_{Li}\gamma^{\mu}W_{\mu}^{+}(V_{CKM})_{ij}q_{Lj}
  + h.c.  
  \label{eq:CCLagrangian}
\end{equation}
\noindent Here, $W_{\mu}^{+}$ is the W-boson field, which couples to the
left-handed quark triplets, $\overbar{q}_{L}^{i}$ and $q_{L}^{j}$, where
$i,j=1,2,3$ are the generation numbers, and $V_{CKM}$ are elements of the
$3\times 3$ Cabbibo-Kobayashi-Maskawa (CKM) matrix~\cite{CKMTheory}. These
elements are coupling constants quantifying the strength of iter- and
intra-generational mixing, and can be represented by complex numbers with arbitrary phases.

For a $3\times 3$ unitary matrix, $3^2=9$ real parameters must be specified. Our
ability to absorb one phase into each quark field, but inability to observe an
overall common phase, removes $5$ of these parameters. $4$ degrees of freedom
therefore remain: $3$ rotations and $1$ complex phase. It is the presence of
this single irreducible phase that can generate $CP$ violation in this sector.

Verifying the unitarity of the CKM matrix is an important test of the
SM~\cite{CKMTheory}. This requirement can be summarised by
$\sum_{k=1}^{3}V_{ik}V^*_{jk}=\delta_{ij}$. Of particular interest is the
condition $V_{ud}V^{*}_{ub}+V_{cd}V^{*}_{cb}+V_{td}V^{*}_{tb}=0$, which forms
the \emph{Unitarity Triangle} (UT) in the complex plane. The geometry of the UT
allows all lengths and angles to be accessed experimentally, and thus it can be
over-constrained, and used as a probe of new physics in precision measurements
of the quark-mixing parameters. 

The angles in this triangle, $\phi_1$, $\phi_2$ and $\phi_3$, can be accessed
experimentally via the study of $B$ meson decays. For many years, the least well
known angle was $\phi_3=\arg(-V_{ud}V^{*}_{ub}/V_{cd}V^{*}_{cb})$ (more commonly
referred to as $\gamma$), but recently its precision now rivals that of
$\phi_2$. $\phi_3$ can be thought of as the Standard Candle of the UT, as it can
be accessed via tree-level decays of $B$ mesons with very low theoretical
uncertainty, $\frac{\delta\phi_3}{\phi_3}\sim 10^{-7}$. The current World
average value using a combination of tree-level measurements is
$\phi_3=(74.0^{+5.0}_{-5.8})$\degree~\cite{LatestGamma}.

It is also possible to constrain $\phi_3$ indirectly, by excluding tree-level
measurements and performing global fits to the CKM triangle, as is shown in
Fig.~\ref{fig:CKMIndirectFit}. The current best indirect measurement is
$\phi_3=(65.64^{+0.97}_{-3.42})$\degree~\cite{website:CKMFitter}. The
difference between the values of the direct and indirect measurements is of
great interest in the current climate. Over the past few years, flavour
anomalies have appeared in measurements of $\mathcal{B} (B^{0}_{s}\rightarrow
\phi \mu \mu)$~\cite{B2phimumu}, $\mathcal{B} (B^{0}\rightarrow K^{*0} \mu
\mu)$~\cite{B2Kstmumu}, $R_{K}$~\cite{Rk}, $R_{K^{*0}}$~\cite{Rkst} and
$R_{D^{(*)-}}$~\cite{RDst}. Though none have reached a significance level to
justify a discovery, some combinations have. These anomalies, and the
contention in the measurements of $\phi_3$, could be explained by the presence
of new, heavy particles. It is therefore motivating to reduce the uncertainty
on $\phi_{3}$ using tree-level measurements to a size comparable to that
extracted from the indirect measurements.
\begin{figure}[H]
	\centering \includegraphics[width=1\textwidth]{figures/CKMIndirectFit.eps}
\caption{\small{Diagram to show current measurements of the Unitarity Triangle
using only constraints from `loop' quantities in the ($\rho$-bar, $\eta$-bar)
plane. The black line represents the fit obtained by CKM fitter.}}
\label{fig:CKMIndirectFit} \vspace{-10pt}
\end{figure}
A significant contribution to this effort comes from studying tree-level
processes of the type $B^{\pm}\rightarrow D^{0(*)}K^{\pm (*)}$~\cite{B2DKD2hh,
DalitzRun1, DalitzRun2, B2DKstD2hh, B2DstKD2hh}. The ADS~\cite{ADSRef} and
GLW~\cite{GLWRef} methods can be employed to access $\phi_3$, the relative weak
phase between the $b\rightarrow c\overbar{u}s$ and $b\rightarrow u\overbar{c}s$
transitions, as seen in the Feynman diagrams in Fig.~\ref{fig:B2DstarKstar}.
This report will study one variant, $B^{\pm}\rightarrow D^{*}K^{\pm}$ decays,
where the neutral D$^{*}$ meson represents $D^{*0}$ or $\overbar{D}^{*0}$ and
is reconstructed as either $D\pi^{0}$ or $D\gamma$. 
% \begin{figure}[H]
% 	\centering
% 	\includegraphics[width=0.8\textwidth]{figures/B2DstarKstar}
% 	\caption{\small{Feynman diagrams for ${B}^{-}\rightarrow{D}^{(*)}{K}^{(*)-}$}.
% The decay to $\overbar{{D}}^{(*)0}$ is CKM
% $|V_{ub}V^{*}_{cs}/V_{cb}V^{*}_{us}|\approx0.4$ and colour ${F}_{CS}\approx1/3$
% suppressed with respect to (w.r.t.) the ${D}^{(*)0}$ final state. The
% interference of these two diagrams is the common mechanism for accessing
% $\phi_3$.}
% 	\label{fig:B2DstarKstar}
% 	\vspace{-10pt}
% \end{figure}

\begin{figure}[H]
\centering
\begin{tikzpicture}[scale=0.8]
\node[anchor=south west,inner sep=0] (image) at (0.5,0)
  {\includegraphics[width=0.8\textwidth]{figures/B2DstarKstar}};
\begin{scope}[x={(image.south east)},y={(image.north west)}]
\node[] at (0.21,1) {(a)};
\node[] at (0.71,1) {(b)};
\end{scope}
\end{tikzpicture}
	\caption{\small{Feynman diagrams for ${B}^{-}\rightarrow{D}^{(*)}{K}^{(*)-}$}.
The decay to $\overbar{{D}}^{(*)0}$ is CKM
$|V_{ub}V^{*}_{cs}/V_{cb}V^{*}_{us}|\approx0.4$ and colour ${F}_{CS}\approx1/3$
suppressed with respect to (w.r.t.) the ${D}^{(*)0}$ final state. The
interference of these two diagrams is the common mechanism for accessing
$\phi_3$.}
	\label{fig:B2DstarKstar}
	\vspace{-10pt}
\end{figure}

\noindent The amplitude for producing a neutral ${D}^*$ meson in the decay $B^-\rightarrow \tilde{D^*}K^-$ can be expressed as:
  \begin{equation}
    \tilde{D}^*=D^{*0}+r_Be^{i(\delta_B-\phi_3)}\overbar{D}^{*0}
    \label{eq:DstTilde}
  \end{equation}
\noindent Writing this in terms of odd and even $CP$ eigenstates, where
$D^{*}_{+}=\frac{D^{*0}+\overbar{D}^{*0}}{\sqrt{2}}$ and
$D^{*}_{-}=\frac{D^{*0}-\overbar{D}^{*0}}{\sqrt{2}}$:
  \begin{equation}
    \tilde{D}^*=\frac{D^{*}_{+}+D^{*}_{-}}{\sqrt{2}}+r_Be^{i(\delta_B-\phi_3)}\frac{D^{*}_{+}-D^{*}_{-}}{\sqrt{2}}
    \label{eq:DstTildeCP}
  \end{equation}
\noindent Here, $r_B=\frac{|A(B^{-}\rightarrow
\overbar{D}^{*0}K^-)|}{|A(B^{-}\rightarrow {D}^{*0}K^-)|}$ and $\delta_{B}$ is
the relative strong phase between these $B^-$ decays.
 
The $CP$ eigenvalue of the $D^*$ state is given by the product
$\lambda_{D^*}=\lambda_D\times \lambda_{\pi^0\text{/}\gamma} \times(-1)^l$. In
the case of the strong $D^*$ decay via $\pi^0$ emission, $\lambda_{\pi^0}=-1$
and $l=1$ therefore $\lambda_{D^*}=\lambda_D$ and $D^*_{\pm}\rightarrow
D_{\pm}\pi^0$. Where as, for $\gamma$ emission, $\lambda_{\gamma}=+1$ and
$l=1$, to conserve parity, therefore $\lambda_{D^*}=-\lambda_D$ and
$D^*_{\pm}\rightarrow D_{\mp}\gamma$.  This introduces a phase shift of $\pi$
($e^{i\pi}=-1$) between the two neutral $D$ mesons produced in the decay
$B^-\rightarrow \tilde{D}^*K^-$ in the $D\gamma$ mode. This logic was first
shown in~\cite{ADSDstar}.
  \begin{equation}
		\tilde{D^*}\rightarrow \tilde{D}\pi^0 :
\tilde{D}=D^0+r_Be^{i(\delta_B-\phi_3)}\overbar{D}^0 \label{eq:DTildePi0}
  \end{equation}
  \begin{equation}
		\tilde{D^*}\rightarrow \tilde{D}\gamma :
\tilde{D}=D^0+r_Be^{i(\delta_B+\pi-\phi_3)}\overbar{D}^0 \label{eq:DTildeGamma}
  \end{equation}
\begin{figure}[H]
  \centering
	\begin{tikzpicture}[scale=1.0]
	\node[anchor=south west,inner sep=0] (image) at (0.5,0)
		{\includegraphics[width=1.0\textwidth]{figures/B2DstKDiagram.png}};
	\begin{scope}[x={(image.south east)},y={(image.north west)}]
	% \node[] at (0.25,0.5) {(a)};
	% \node[] at (0.75,0.5) {(b)};
	\end{scope}
	\end{tikzpicture}
  \caption{\small{Decay diagram depicting ${B}^{-}\rightarrow{D}^{*}{K}^{-}$ to
  a general $D^{*}$ final state, $f(D^{*})$, and the complex conjugate decay.
  These decays proceed via 2 interfering amplitudes, $D^{*}$ or 
  $\overbar{D}^{*}$ states.}}
\label{fig:B2DstKDiagram} \vspace{-10pt}
\end{figure}
Labelling the amplitude of the diagram displayed in Fig.~\ref{fig:B2DstarKstar} (a)
as $A_{B}=A(B^{-}\rightarrow
D^{*0}K^{-})=A(B^{+}\rightarrow \overbar{D}^{*0}K^{+})$, as shown in Fig.~\ref{fig:B2DstKDiagram}, we can construct 4
equations for the decay amplitudes of $B^{-}$ mesons to final states
$f(\accentset{\brobor}{D}\phantom{})$:
\begin{equation}
    A_{Bf}^{\pi^{0}}=A(B^{-}\rightarrow
    [f(D)\pi^{0}]_{D^{*}}K^{-})=A_{B}A_{\pi^{0}}(A_{D}+\overbar{A}_{D}e^{-i\delta_D}r_{B}e^{(i(\delta_{B}-\phi_{3}))})
    \label{eq:A(Bf)pi0}
  \end{equation}
  \begin{equation}
		A_{B\overbar{f}}^{\pi^{0}}=A(B^{-}\rightarrow
[f(\overbar{D})\pi^{0}]_{D^{*}}K^{-})=A_{B}A_{\pi^{0}}(\overbar{A}e^{-i\delta_D}_{D}+A_{D}r_{B}e^{(i(\delta_{B}-\phi_{3}))})
\label{eq:A(Bfbar)pi0}
  \end{equation}
  \begin{equation}
    A_{Bf}^{\gamma}=A(B^{-}\rightarrow
    [f(D)\gamma]_{D^{*}}K^{-})=A_{B}A_{\gamma}(A_{D}+\overbar{A}_{D}e^{-i\delta_D}r_{B}e^{(i(\delta_{B}+\pi-\phi_{3}))})
    \label{eq:A(Bf)gamma}
  \end{equation}
  \begin{equation}
		A_{B\overbar{f}}^{\gamma}=A(B^{-}\rightarrow
[f(\overbar{D})\gamma]_{D^{*}}K^{-})=A_{B}A_{\gamma}(\overbar{A}_{D}e^{-i\delta_D}+A_{D}r_{B}e^{(i(\delta_{B}+\pi-\phi_{3}))})
\label{eq:A(Bfbar)gamma}
  \end{equation}
\noindent Compared to Fig.~\ref{fig:B2DstKDiagram}, in these equations we have
replaced $\accentset{\brobor}{A}\phantom{}_{D^{*}}$ with
$A_{\pi^{0}}\accentset{\brobor}{A}\phantom{}_{D}$ or
$A_{\gamma}\accentset{\brobor}{A}\phantom{}_{D}e^{{i\pi}}$. These amplitudes
are defined as: $\overbar{A}_{D}=A(D^{0}\rightarrow
f(\overbar{D}))=A(\overbar{D}^{0}\rightarrow f(D))$ and
$A_{D}=A(\overbar{D}^{0}\rightarrow f(\overbar{D}))=A(D^{0}\rightarrow f(D))$.
$\delta_D$ is the relative strong phase difference between the $D$ and
$\overbar{D}$ mesons decaying to the same final state. 4 analogous charge
conjugate equations can be written for the $B^{+}$ meson. 

When applying GLW method~\cite{GLWRef}, the $D$ meson is reconstructed in the
$CP$-even final states $D\rightarrow KK$ and $D\rightarrow \pi \pi$, therefore
$\delta_D=0$ and $A_{D}=\overbar{A}_{D}$ can also be factorised out. Time integrated decay rates are proportional to the squared magnitude of the amplitudes, therefore we can construct the following equations for $B^{\pm}$ mesons:
  \begin{equation}
		\Gamma_{CP+}^{\pm}(B^{\pm}\rightarrow (D^{*}\rightarrow
[hh]_D\pi^{0}))\propto |A^{\pi^0}_{Bf_{CP}}|^2 \propto 1 + r_{B}^{2} +
2r_{B}\cos(\delta_{B}\mp \phi_{3}) \label{eq:DecayRateGLWpi0}
  \end{equation}
  \begin{equation}
		\Gamma_{CP-}^{\pm}(B^{\pm}\rightarrow (D^{*}\rightarrow
[hh]_D\gamma))\propto |A^{\gamma}_{Bf_{CP}}|^2 \propto 1 + r_{B}^{2} -
2r_{B}\cos(\delta_{B}\mp \phi_{3}) \label{eq:DecayRateGLWgamma}
  \end{equation}
\noindent The strong phase shift of $\pi$ from the
$D^{*}\rightarrow D\gamma$ decay converts the $CP+$ eigenstate from even to odd
in the context of $B^{\pm}\rightarrow D^{(*)0}K^{\pm}$ decays. This provides a
useful extra constraint in the extraction of $\phi_3$. 

In the ADS method~\cite{ADSRef}, the $D$ meson is reconstructed as
$D^{0}\rightarrow K^{+}\pi^{-}$, $D^{0}\rightarrow K^{-}\pi^{+}$ and their
charge conjugates. The former is doubly Cabbibo suppressed with respect to the
latter, therefore we label these as the suppressed (SUP) and favoured (FAV)
modes. Defining $A_{D}=A(D^{0}\rightarrow
K^{-}\pi^{+})=A(\overbar{D}^{0}\rightarrow K^{+}\pi^{-})$ and $\overbar{A}_{D}=
A(D^{0}\rightarrow K^{+}\pi^{-})=A(\overbar{D}^{0}\rightarrow
K^{-}\pi^{+})=A_{D}r_{D}e^{-i\delta_{D}}$, where $r_{D}$ is the amplitude ratio
of the SUP with respect to the FAV $D$ decay mode and $\delta_{D}$ is their
relative phase, we can construct the following decay rate equations for
$B^{\pm}$ mesons:
  \begin{align*}
		\Gamma_{SUP}^{\pm}(B^{\pm}\rightarrow (D^{*}\rightarrow
[K^{\mp}\pi^{\pm}]_D\pi^{0})) &\propto |A^{\pi^0}_{Bf_{SUP}}|^2 \propto r_{B}^{2}
+ r_{D}^{2} + 2r_{B}r_{D}\cos(\delta_{B} - \delta_{D} \mp \phi_{3}) \\
		\Gamma_{SUP}^{\pm}(B^{\pm}\rightarrow (D^{*}\rightarrow
[K^{\mp}\pi^{\pm}]_D\gamma)) &\propto |A^{\gamma}_{Bf_{SUP}}|^2 \propto r_{B}^{2}
+ r_{D}^{2} - 2r_{B}r_{D}\cos(\delta_{B} - \delta_{D} \mp \phi_{3}) \\
		\Gamma_{FAV}^{\pm}(B^{\pm}\rightarrow (D^{*}\rightarrow
[K^{\pm}\pi^{\mp}]_D\pi^{0})) &\propto |A^{\pi^0}_{Bf_{FAV}}|^2 \propto 1 +
r_{B}^{2}r_{D}^{2} + 2r_{B}r_{D}\cos(\delta_{B} + \delta_{D} \mp \phi_{3}) \\
		\Gamma_{FAV}^{\pm}(B^{\pm}\rightarrow (D^{*}\rightarrow
[K^{\pm}\pi^{\mp}]_D\gamma)) &\propto |A^{\gamma}_{Bf_{FAV}}|^2 \propto 1 +
r_{B}^{2}r_{D}^{2} - 2r_{B}r_{D}\cos(\delta_{B} + \delta_{D} \mp \phi_{3})
  \end{align*}
\noindent The SUP mode is known as the ADS mode. $r_{D}$ and $r_{B}$ are of
similar magnitudes so the ADS mode possesses particularly high sensitivity to
$\phi_3$ as the relative size of the interference term is large. In the usual
ADS analysis~\cite{ADSRef}, two independent equations are obtained with three
unknowns, $r_B$, $\delta_B$ and $\phi_3$ ($r_{D}$ and $\delta_D$ are known from
measurements of $D$ decays). Performing the same method using
$B^{\pm}\rightarrow D^{*}K^{\pm}$ decays results in four independent equations. This
enhanced ADS method therefore has the potential to measure $\phi_3$ without
information from other modes.

\section{\normalsize $CP$ Observables} \label{sec:observables}
We create observables that do not depend on absolute efficiencies by taking
ratios. The physics observables that will be measured in this analysis are:

\begin{enumerate}
  \item The ratio of $B^{\pm} \rightarrow D^* K^{\pm}$ to $B^{\pm} \rightarrow
  D^* \pi^{\pm}$ decays for the favoured $D$ decay mode:
  \begin{equation}
    R^{K\pi , \pi^0 /\gamma}_{K/\pi}=\frac{\Gamma(B^{-}\rightarrow
    ([K^{-}\pi^{+}]_D\pi^0 /\gamma)_{D^*}K^{-})+\Gamma(B^{+}\rightarrow
    ([K^{+}\pi^{-}]_D\pi^0 /\gamma)_{D^*}K^{+})}{\Gamma(B^{-}\rightarrow
    ([K^{-}\pi^{+}]_D\pi^0 /\gamma)_{D^*}\pi^{-})+\Gamma(B^{+}\rightarrow
    ([K^{+}\pi^{-}]_D\pi^0 /\gamma)_{D^*}\pi^{+})}
	\label{eq:R_DstK_Dstpi}
  \end{equation}
  \item The $CP$ asymmetries between $B^{-}$ and $B^{+}$ mesons decaying to $CP$
  eigenstates, where $hh=\pi \pi /KK$:
		\begin{equation}
      A_{hh}^{\pi^{0}/\gamma}=\frac{\Gamma(B^{-}\rightarrow
      [[h^{+}h^{-}]_{D}\pi^{0}/\gamma]_{D^{*}}K^{-})-\Gamma(B^{+}\rightarrow
      [[h^{+}h^{-}]_{D}\pi^{0}/\gamma]_{D^{*}}K^{+})}{\Gamma(B^{-}\rightarrow
      [[h^{+}h^{-}]_{D}\pi^{0}/\gamma]_{D^{*}}K^{-})+\Gamma(B^{+}\rightarrow
      [[h^{+}h^{-}]_{D}\pi^{0}/\gamma]_{D^{*}}K^{+})} \label{eq:Aglw}
		\end{equation}
  \item The ratio of $B^{\pm} \rightarrow D^* K^{\pm}$ to $B^{\pm} \rightarrow
  D^* \pi^{\pm}$ decays for $D\rightarrow hh$, where $hh=\pi \pi /KK$, divided
  by $R^{K\pi , \pi^0 /\gamma}_{K/\pi}$:
		\begin{equation}
      R_{hh}^{\pi^{0}/\gamma}=\frac{\Gamma(B^{-}\rightarrow
      ([h^{-}h^{+}]_D\pi^0 /\gamma)_{D^*}K^{-})+\Gamma(B^{+}\rightarrow
      ([h^{+}h^{-}]_D\pi^0 /\gamma)_{D^*}K^{+})}{\Gamma(B^{-}\rightarrow
      ([h^{-}h^{+}]_D\pi^0 /\gamma)_{D^*}\pi^{-})+\Gamma(B^{+}\rightarrow
      ([h^{+}h^{-}]_D\pi^0 /\gamma)_{D^*}\pi^{+})} \times \frac{1}{R^{K\pi ,
      \pi^0 /\gamma}_{K/\pi}}
		\end{equation}
  \item The ratios of the ADS to the favoured mode, for $B^{\pm}$ decays. We
  measure $R_{\pm}$ as opposed to $A_{ADS}$ and $R_{ADS}$ because the latter are
  statistically correlated. It should be noted that this constitutes 4
  independent equations.
		\begin{equation}
			R_{\pm}^{\pi^{0}/\gamma}=\frac{\Gamma(B^{\pm}\rightarrow
			[[K^{\mp}\pi^{\pm}]_{D}\pi^{0}/\gamma]_{D^{*}}K^{\pm})}{\Gamma(B^{\pm}\rightarrow
			[[K^{\pm}\pi^{\mp}]_{D}\pi^{0}/\gamma]_{D^{*}}K^{\pm})} \label{eq:Rads}
		\end{equation}
\end{enumerate}
\noindent Direct $CP$ violation in $D$ decays is considered negligible, therefore
$A_{KK}^{\pi^{0}}=A_{\pi\pi}^{\pi^{0}}=A_{CP+}$,
$A_{KK}^{\gamma}=A_{\pi\pi}^{\gamma}=A_{CP-}$,
$R_{KK}^{\pi^{0}}=R_{\pi\pi}^{\pi^{0}}=R_{CP+}$ and
$R_{KK}^{\gamma}=R_{\pi\pi}^{\gamma}=R_{CP-}$.

We can express these ratios in terms of the parameters of interest:
  \begin{equation}
    R^{K\pi,
    \pi^0}_{K/\pi}=R\frac{1+(r_Br_D)^2+2r_Br_D\cos(\delta_B-\delta_D)\cos(\phi_3)}{1+(r_B^{\pi}r_D)^2+2r_B^{\pi}r_D\cos(\delta_B^{\pi}-\delta_D)\cos(\phi_3)}
    \label{eq:Rkpipi0}
  \end{equation}
  \begin{equation}
    R^{K\pi,
    \gamma}_{K/\pi}=R\frac{1+(r_Br_D)^2-2r_Br_D\cos(\delta_B-\delta_D)\cos(\phi_3)}{1+(r_B^{\pi}r_D)^2-2r_B^{\pi}r_D\cos(\delta_B^{\pi}-\delta_D)\cos(\phi_3)}
    \label{eq:Rkpigamma}
  \end{equation}
  \begin{equation}
    A_{CP\pm}=\frac{\pm 2r_{B}\sin(\delta_{B})\sin(\phi_{3})}{1+r_{B}^{2}\pm 2r_{B}\cos(\delta_{B})\cos(\phi_{3})}
    \label{eq:ACP}
  \end{equation}
  \begin{align}
    R_{CP\pm} &=R\frac{1 + r_{B}^{2} \pm 2r_{B}\cos(\delta_{B})\cos(\phi_{3})}{1 +
    r_{B}^{\pi^2} \pm 2r_{B}^{\pi}\cos(\delta_{B}^{\pi})\cos(\phi_{3})} \times
    \frac{1}{R^{K\pi,\pi^0 /\gamma}_{K/\pi}} \nonumber \\
    &= 1 + r_B^2 + 2r_B\cos(\delta_B)\cos(\phi_3) 
    \label{eq:RCP}
  \end{align}
  \begin{equation}
    A_{K\pi}^{\pi^{0}}=\frac{2r_{B}\sin(\delta_{B}+\delta_{D})\sin(\phi_{3})}{1+r_{B}^{2}r_{D}^{2}+2r_{B}r_{D}\cos(\delta_{B}+\delta_{D})\cos(\phi_{3})}
    \label{eq:AfavPi0}
  \end{equation}
  \begin{equation}
    R_{\pm}^{\pi^{0}}=\frac{r_{B}^{2} + r_{D}^{2} +
    2r_{B}r_{D}\cos(\delta_{B}-\delta_{D}\mp
    \phi_{3})}{1+r_{B}^{2}r_{D}^{2}+2r_{B}r_{D}\cos(\delta_{B}+\delta_{D}\mp
    \phi_{3})} \label{eq:RPlusMinusPi0}
  \end{equation}
  \begin{equation}
    A_{K\pi}^{\gamma}=\frac{2r_{B}\sin(\delta_{B}+\delta_{D})\sin(\phi_{3})}{1+r_{B}^{2}r_{D}^{2}-2r_{B}r_{D}\cos(\delta_{B}+\delta_{D})\cos(\phi_{3})}
    \label{eq:AfavGamma}
  \end{equation}
  \begin{equation}
    R_{\pm}^{\gamma}=\frac{r_{B}^{2} + r_{D}^{2} -
    2r_{B}r_{D}\cos(\delta_{B}-\delta_{D}\mp
    \phi_{3})}{1+r_{B}^{2}r_{D}^{2}-2r_{B}r_{D}\cos(\delta_{B}+\delta_{D}\mp
    \phi_{3})} \label{eq:RPlusMinusGamma}
  \end{equation}

\noindent In Eq.~\eqref{eq:Rkpipi0}, \eqref{eq:Rkpigamma} and \eqref{eq:RCP},
$R=\frac{B^{\pm}\rightarrow D^* K^{\pm}}{B^{\pm}\rightarrow D^* \pi^{\pm}}$,
$r_B^{\pi}=\frac{|A(B^{-}\rightarrow
\overbar{D}^{*0}\pi^-)|}{|A(B^{-}\rightarrow {D}^{*0}\pi^-)|}$ and
$\delta_{B}^{\pi}$ is the relative strong phase between these $B$ decays. To
arrive at Eq.~\eqref{eq:RCP}, it was approximated that $r_B^{\pi}\ll 1$ and
$r_Br_D \ll 1$.

For $B^{\pm}\rightarrow D^{*}K^{\pm}$ decays, the
current values for $r_B$ and $\delta_B$ are $0.119^{+0.018}_{-0.019}$ and
$(-49^{+12}_{-15})$\degree, respectively~\cite{website:CKMFitter}. GLW analyses
of $B^{\pm}\rightarrow D^{*}K^{\pm}$ decays have been successfully studied by
the Belle~\cite{BelleGLW}, BaBar~\cite{BaBarGLW} and LHCb~\cite{PartReco}
collaborations. The latter was performed recently using a partially
reconstructed analysis, where the neutral particle was not included in the final
state. No results have yet been published on the ADS mode from the LHCb
collaboration.   
% In 2014, the BaBar experiment attempted the ADS analysis, but in the absence
% of significant signal set the upper limit $r_B<0.21$ ($90\%$ C.L.)
% \cite{BaBarADS}. 

In the following report, $B^{\pm}\rightarrow D^{*}\pi^{\pm}$ decays have also
been studied. These decays are almost kinetically identical to
$B^{\pm}\rightarrow D^{*}K^{\pm}$ decays but have a larger branching fraction.
The higher statistics have been exploited when developing the event selection
and mass fit for this analysis, however the value of $r_B$ for $D^*\pi$ final
states is $\sim 5\%$ that of $D^*K$ final states, therefore decays of this
type are less sensitive to $\phi_3$.

\section{The LHCb Detector and Neutral Particle Reconstruction} \label{sec:detector}
The LHCb~\cite{LHCbDetector} detector is located at the Large Hadron Collider, a
27 km circular $pp$ collider located at CERN. LHCb has been designed to study
$b$ and $c$ quarks, which are predominantly produced in the forward/backward
direction from gluon fusions into $c\overbar{c}$ or $b\overbar{b}$ pairs. For
this reason, LHCb is a forward spectrometer covering the pseudorapidity region
$2 < \eta < 5$, where $\eta = -\ln (\tan \theta / 2)$.  Here, $\theta$ is the
angle between the particle's momentum vector and the beam axis. During Run1
(Run2), the LHC operated with a collision frequency of approximately 20 MHz (40
MHz). At the interaction point, the transverse overlap of the 2 beams is reduced
so that, on average, 1.6 (1.1) $pp$ interactions occur per bunch crossing in
Run1 (Run2). To reduce the rate of data uptake, online event selection consists
of a hardware and software trigger. The hardware trigger uses information from
the calorimeter and muon systems, and reduces the rate of information that needs
to be stored down to 1 MHz, allowing latency for the full detector to be read
out. This is required for the software trigger, which involves a full event
reconstruction.  Events that pass the software trigger are read out at a rate of
3 kHz (2011), 5 kHz (2012) and 12.5 kHz (Run2).

The full description of the LHCb detector can be found in \cite{LHCbDetector},
whilst the relevant components will be described here.  The vertex locator, a
semiconductor silicon detector with a transverse resolution of $\mathcal{O}(10$
$\mu$m), sits $8$ mm from the beam line in order to identify the $B$ and $D$
spatial decay vertices crucial for this analysis. The rest of the tracking
system is located further from the $pp$ interaction point and consists of the
Tracker Turicensis (TT), positioned upstream of the dipole magnet, and tracking
stations T1-T3, downstream of the magnet. The TT and the inner tracker (IT) of
T1-T3 are silicon microstrip detectors, whilst the outer tracker (OT) of the
tracking stations consists of straw drift tubes. The dipole magnet itself has an
integrated magnetic field $\int B dl = 4$ Tm.  Positive and negatively charged
particles bend in opposite directions in this field, therefore, to avoid charge
detection asymmetries from small differences in the left and right parts of the
detector, the magnetic field is reversed regularly during data taking.  There
are also two Ring Imaging Cherenkov detectors (RICH1 and RICH2), placed either
side of the magnet, to provide direct discrimination between kaons, pions and
protons.

The reconstruction of neutrals involves the calorimeter system, consisting of
the electromagnetic (ECAL) and hadronic (HCAL) calorimeters preceded by the
Silicon Pad Detector (SPD) and Preshower (PS), separated by a lead wall. The
ECAL consists of alternating scintillating tiles and lead plates, and the HCAL
iron plates interspaced with scintillating tiles aligned parallel to the beam
pipe. The SPD identifies whether incident particles are charged or neutral,
whilst the PS determines whether they are electrons (if charged) or photons (if
neutral), by looking at the longdtudinal segmentation of the electromagnetic
shower, in order to discriminate against pions. Both the SPD and PS consist of
scintillating pads and all light produced in the system is sent down
wavelength-shifting fibres and transmitted to photomultiplier tubes.

Once a candidate has been identified as a $\gamma$ or $\pi^0$ from its energy
deposition sequence in the calorimeter system, reconstruction is performed using
the ECAL \cite{NeutralReconstruction}. Energy deposits in ECAL cells are grouped
together into clusters by applying a $3\times3$ cell pattern around the local
energy deposition maxima. If clusters overlap, a convergent iterative procedure
is applied that redistributes the cell energy between the clusters
proportionally to the total cluster energies.

Photon candidates are identified to be neutral clusters without any associated
extrapolated track. This means that they must have a $\chi^2_{2D}$, defined
below, of at least 4 for every track in the event.
\begin{equation}
  \chi^2_{2D}=(\vec{r}_{tr}-\vec{r}_{cl})^T(C_{tr}+S_{cl})^{-1}(\vec{r}_{tr}-\vec{r}_{cl})
  \label{Chi2}
\end{equation}
\noindent Here, $\vec{r}_{tr}$ and $\vec{r}_{cl}$ represent the 2D coordinates
of extrapolated tracks and energy-weighted cluster centres, respectively.
$C_{tr}$ is the covariance matrix of the $\vec{r}_{tr}$ parameters and $S_{cl}$
is the $2 \times 2$ cluster energy spread matrix.  

The photon energy is then the sum of the total cluster energy in the ECAL and
the reconstructed energy deposit in the PS. Its direction is determined using
the assumed origin and cluster centre of the candidate. It should be noted that,
in this report, only unconverted photons have been considered, i.e. those that
have pair produced after passing through the SPD.

Neutral pions are reconstructed as pairs of well-separated photons. These
$\pi^0$s are required to have transverse momentum ($p_T$) in the range $p_{T}<2$
GeV/c.  Above this, the daughter photons are not sufficiently resolved as
individual clusters due to finite ECAL granularity. Neutral pion candidates
where the photon clusters are merged are not considered in this analysis.

A charged particle passing through the whole tracking system, with momentum
withn the range 5 GeV/$c$ $< p <$ 200 GeV/$c$, has a 96\% probability of being
reconstructed. In contrast, photons and $\pi^0$ mesons have a reconstruction
efficiency of around 10\% and 3\%, respectively.

The data presented in this report consists of $6.6$ $\text{fb}^{-1}$ of $pp$
collision data, broken down by center-of-mass energies into $1$ $\text{fb}^{-1}$ taken at $7$ TeV in 2011, $2$
$\text{fb}^{-1}$ at $8$ TeV in 2012 (Run1) and $3.6$ $\text{fb}^{-1}$ at $13$
TeV in 2015, 2016 and 2017 (Run2). The increase in energy between Run1 and Run2
resulted in an increase in the $b\overbar{b}$ production cross
section~\cite{PDG2018}, which, when combined with improvements in the online
selection, caused the rate of selected signal candidates per integrated
luminosity to increase by approximately a factor of 2.

\section{Selection of $B^{\pm}\rightarrow D^*h^{\pm}$ Candidates} \label{sec:selections}

Signal candidates are initially identified by selecting combinations of $D^0$
candidates and a charged track ($h^{\pm}$) within the invariant mass range 4700
$<$ m($D^0h$) $<$ 7200 MeV/$c^2$. A neutral particle from the event is then
combined with the $D$ meson candidate to make a $B^{\pm}\rightarrow D^*h^{\pm}$
candidate. A kinematic fit is performed to the entire decay chain of each
candidate, with vertex constraints applied to both the $B^{\pm}$ and $D$
daughters, and $D$ and $\pi^0$ (when present) mesons are constrained to their
known masses, 1864.8 MeV/$c^2$ and 135.0 MeV/$c^2$ respectively.

Candidates must fulfil the hardware trigger, meaning that decay products of the
signal candidate detected by the HCAL, or events containing at least one
candidate from elsewhere in the event, must lie above a fixed threshold in
transverse energy.  The software trigger must also be passed. This places
requirements on the quality of tracks, which are then combined one-by-one, and
identified as having either 2-, 3-, or 4-body topology depending on their
distance of closest approach. 

The $D$ meson is required to be within $\pm$25 MeV/$c^2$ of the known $D^0$
mass, which corresponds to approximately three times the mass resolution.  When
a neutral pion is present, it is required to lie within the asymmetric mass
window of $110 < m[\pi^0] < 185$ MeV/$c^2$. In order to remove combinatorial
backgrounds from low energy photons in the event, transverse momentum
requirements are placed on the neutral particles. In the $D^*\rightarrow
D\gamma$ final state, $\gamma_{P_T}>350$ MeV/$c^2$. In the $D^*\rightarrow
D\pi^0$ mode, $\pi^0_{P_T}>350$ MeV/$c^2$, and the $\pi^0$ secondary photons are
required to have $P_T>200$ MeV/$c^2$.

Particle identification (PID) information is obtained from the RICH detector.
The charged pion and kaon from the $B^{\pm}$ decay vertex are required to occupy
the momentum regions $5 < p < 100$ GeV/$c$ and $0.5 < p_T < 10$ GeV/$c$, to lie
within the RICH's kinematic range. PID cuts are placed on these particles, and
on the $D$ meson daughters, to identify them as pions or kaons. This prevents
the charged meson from the $B^{\pm}$ decay vertex and $D$ meson candidates from
appearing in more than one category. Cross feed between the final states
$D^{0}\rightarrow K^-\pi^+, K^-K^+, \pi^-\pi^+$ is negligible. When both $D$
secondaries are misidentified, however, there is some contamination of
Cabbibo-favoured $D^0\rightarrow K^-\pi^+$ decays in the $D^0\rightarrow
K^+\pi^-$ sample, due to its relatively large branching fraction. To suppress
this crossfeed, a veto is applied to $D$ final states containing a pion and
kaon.  The $D$ candidate is reconstructed with the mass hypothesis of the
daughters swapped, $i.e.$ the kaon is reconstructed as a pion and the pion is
reconstructed as a kaon. These mass-swapped $D$ candidates that lie within 15
MeV/$c^2$ of the nominal $D$ mass are removed. 

Combinatorial backgrounds are suppressed using two stages of multivariate
analysis (MVA). During the first stage, fake $D$ candidates, formed from the
combination of two random tracks, and fake $B$ candidates, made up of a $D$
meson plus a random track, are targeted.  The use of Boosted Decision Trees
(BDTs)~\cite{RegressionTrees} in isolating $B^{\pm}\rightarrow
[h^{\pm}h^{\mp}]_{D}h^{\pm}$ decays has been well established by an earlier 
ADS/GLW analysis of this mode~\cite{B2DKD2hh}. A similar strategy is employed
here. 

\begin{figure}[H]
\centering
\begin{tikzpicture}[scale=0.48]
\node[anchor=south west,inner sep=0] (image) at (0,0) {\includegraphics[width=0.48\textwidth]{figures/D0hMvsDeltaMpi0}};
\begin{scope}[x={(image.south east)},y={(image.north west)}]
% \draw[help lines,xstep=.1,ystep=.1] (0,0) grid (1,1);
% \foreach \x in {0,1,...,9} { \node [anchor=north] at (\x/10,0) {0.\x}; }
% \foreach \y in {0,1,...,9} { \node [anchor=east] at (0,\y/10) {0.\y}; }
\draw[green,ultra thick] (0.533,0.0899) rectangle (0.84,0.899);
\node[green,ultra thick] at (0.7,0.5) {BDT1};
\draw[red,ultra thick] (0.2,0.355) rectangle (0.39,0.899);
\node[red,ultra thick] at (0.295,0.62) {BDT2};
\end{scope}
\end{tikzpicture}
\begin{tikzpicture}[scale=0.48]
\node[anchor=south west,inner sep=0] (image) at (0.5,0)
  {\includegraphics[width=0.48\textwidth]{figures/D0hMvsDeltaMgamma}};
\begin{scope}[x={(image.south east)},y={(image.north west)}]
% \draw[help lines,xstep=.1,ystep=.1] (0,0) grid (1,1);
% \foreach \x in {0,1,...,9} { \node [anchor=north] at (\x/10,0) {0.\x}; }
% \foreach \y in {0,1,...,9} { \node [anchor=east] at (0,\y/10) {0.\y}; }
\draw[green,ultra thick] (0.533,0.12) rectangle (0.84,0.899);
\node[green,ultra thick] at (0.7,0.55) {BDT1};
\draw[red,ultra thick] (0.2,0.499) rectangle (0.39,0.899);
\node[red,ultra thick] at (0.295,0.7) {BDT2};
\end{scope}
\end{tikzpicture}
\caption{\small{2D histogram of $\Delta_M$ plotted against $m[D\pi]$. The
training regions in $B^{\pm}\rightarrow (D^*\rightarrow
[K^{\pm}\pi^{\mp}]_D\pi^0)\pi^{\pm}$ (left) and $B^{\pm}\rightarrow
(D^*\rightarrow [K^{\pm}\pi^{\mp}]_D\gamma)\pi^{\pm}$ (right) data for both
stages of MVA are shown by the labelled boxes.}} \label{fig:trainingData}
\end{figure}

Data side-bands for the high statistics modes $B^{\pm}\rightarrow(D^{*}
\rightarrow [K^{\pm}\pi^{\mp}]_{D} \pi^0 /\gamma)\pi^{\pm}$ are provided as a
background sample for the BDT. Here, side-bands constitutes data that falls in
the range 5800 $ < m[D\pi] <$ 6800 MeV/$c^2$, as shown in
Fig.~\ref{fig:trainingData}, where $\Delta_M=m[D^{*}] - m[D]- (m[\pi^0] +
m[\pi^0]_{PDG})$ (the part in brackets refers to the definition for the
$D^*\rightarrow D\pi^0$ decay mode). The signal sample consists of simulated
events of the same decay mode. The training variables used are related to the
decay kinematics and topology of tracks, and are displayed in
Table~\ref{table:bdt1TrainingVar}. 

The working point of BDT1 $>$ 0.05 is chosen by performing fits to the $D$ mass
distribution of the high statistics mode, in BDT steps of 0.05. The extracted
signal yield is then scaled to the ADS mode using the branching ratio
$\mathcal{BR} = \frac{\mathcal{BF}(D^0 \rightarrow K^- \pi^+)}{\mathcal{BF}(D^0
\rightarrow K^+ \pi^-)}$. The optimisation was performed using the significance
figure of merit, $\sigma$, defined below, where $S$ and $B$ represent the signal
and background yields, respectively.
\begin{equation}
\sigma = \frac{S}{S \times B}
\label{D_branching_ratio}
\end{equation}

BDTs are also employed in the second stage MVA, designed to select the true
neutral and $D^*$ candidate. Separate BDTs are trained for the two $D^*$ final
states, $D\pi^0$ and $D\gamma$. The signal samples provided are made up of
$B^{\pm}\rightarrow (D^*\rightarrow [K^{\pm}\pi^{\mp}]\pi^0/\gamma)\pi^{\pm}$
simulation samples. The background samples consist of data in the
$\Delta_M=m[D^{*}] - m[D]- (m[\pi^0] + m[\pi^0]_{PDG})$ upper side-bands, 250 $
< \Delta_{M} < $ 500 MeV/$c^2$, but $m[D\pi]$ signal region, as shown in
Fig.~\ref{fig:trainingData}. The training variables, listed in
Table~\ref{table:bdt2TrainingVar}, are a mixture of momentum variables and
parameters quantifying photon quality. 

The same optimisation procedure is used for the second stage BDT, with signal
and background yields extracted from fits to $\Delta_M$ distributions. The
working point chosen to optimise $\sigma$ is BDT2 $>0.05$ for both $D\pi^0$ and
$D\gamma$ final states, which has a signal efficiency of 40\% and 50\%,
respectively.
\begin{figure}[H]
\centering
\begin{tikzpicture}[scale=0.24]
\node[anchor=south west,inner sep=0] (image) at (0.5,0)
  {\includegraphics[width=0.24\textwidth]{figures/DstNeutCLGammaMC}};
\end{tikzpicture}
\begin{tikzpicture}[scale=0.24]
\node[anchor=south west,inner sep=0] (image) at (0.5,0)
  {\includegraphics[width=0.24\textwidth]{figures/DstNeutIsNotEGammaMC}};
\end{tikzpicture}
\begin{tikzpicture}[scale=0.24]
\node[anchor=south west,inner sep=0] (image) at (0.5,0)
  {\includegraphics[width=0.24\textwidth]{figures/DstNeutPTGammaMC}};
\end{tikzpicture}
\begin{tikzpicture}[scale=0.24]
\node[anchor=south west,inner sep=0] (image) at (0.5,0)
  {\includegraphics[width=0.24\textwidth]{figures/Dstar0PTGammaMC}};
\end{tikzpicture}
\begin{tikzpicture}[scale=0.24]
\node[anchor=south west,inner sep=0] (image) at (0.5,0)
  {\includegraphics[width=0.24\textwidth]{figures/DstNeutCLGammaData}};
\begin{scope}[x={(image.south east)},y={(image.north west)}]
\node[] at (0.5, -0.1) {(a)};
\end{scope}
\end{tikzpicture}
\begin{tikzpicture}[scale=0.24]
\node[anchor=south west,inner sep=0] (image) at (0.5,0)
  {\includegraphics[width=0.24\textwidth]{figures/DstNeutIsNotEGammaData}};
\begin{scope}[x={(image.south east)},y={(image.north west)}]
\node[] at (0.5, -0.1) {(b)};
\end{scope}
\end{tikzpicture}
\begin{tikzpicture}[scale=0.24]
\node[anchor=south west,inner sep=0] (image) at (0.5,0)
  {\includegraphics[width=0.24\textwidth]{figures/DstNeutPTGammaData}};
\begin{scope}[x={(image.south east)},y={(image.north west)}]
\node[] at (0.5, -0.1) {(c)};
\end{scope}
\end{tikzpicture}
\begin{tikzpicture}[scale=0.24]
\node[anchor=south west,inner sep=0] (image) at (0.5,0)
  {\includegraphics[width=0.24\textwidth]{figures/Dstar0PTGammaData}};
\begin{scope}[x={(image.south east)},y={(image.north west)}]
\node[] at (0.5, -0.1) {(d)};
\end{scope}
\end{tikzpicture}
\caption{\small{Distributions of the input variables to the second stage BDT for
the $D^* \rightarrow D\gamma$ decay mode. The top and bottom rows show plots for
simulation and data, respectively. The variables are labelled by letters: (a)
shows the confident level, CL, of the photon, (b) the photon IsNotE, (c) the
photon $p_T$ and (d) the $D^*$ meson $p_T$. The red points show the
distributions before the cut on BDT2, and the blue points show the distributions
after.}} \label{fig:GammaBDT2Variables}
\end{figure}

The input variables have been studied before and after this BDT cut, and their
normalised distributions are show in Fig.~\ref{fig:GammaBDT2Variables} and
Fig.~\ref{fig:Pi0BDT2Variables} for the $D\gamma$ and $D\pi^0$ $D^*$ final
states, respectively. It can be seen that the effect of the BDT it to bring the
data distributions closer to those of simulated signal.  The $p_T$ variables,
higher momenta neutrals are favoured due to the increased combinatorics at low
$p_T$.

Events reconstructed in both $D^{*}$ decay modes modes are kept in the $D\pi^0$
and removed from the $D\gamma$ sample. This is motivated by the reduced
statistics in the $D\pi^0$ final state and an increase in purity of signal data
in the $D\gamma$ sample. Multiple candidates occur at a rate of $\sim 1.5$ in
the $D\pi^0$ mode and $\sim 1.1$ in the $D\gamma$ mode. This reduced rate in the
$D\gamma$ mode is due to the tighter transverse momentum cut on the photons in
the event, removing combinatoric background from low energy photons. The
selected candidate to be used in the mass fit is chosen at random.
\begin{figure}[H]
\centering
\begin{tikzpicture}[scale=0.19]
\node[anchor=south west,inner sep=0] (image) at (0.5,0)
  {\includegraphics[width=0.19\textwidth]{figures/DstNeutCLPi0MC}};
\end{tikzpicture}
\begin{tikzpicture}[scale=0.19]
\node[anchor=south west,inner sep=0] (image) at (0.5,0)
  {\includegraphics[width=0.19\textwidth]{figures/DstNeutIsNotEPi0MC}};
\end{tikzpicture}
\begin{tikzpicture}[scale=0.19]
\node[anchor=south west,inner sep=0] (image) at (0.5,0)
  {\includegraphics[width=0.19\textwidth]{figures/DstNeutPTtotPi0MC}};
\end{tikzpicture}
\begin{tikzpicture}[scale=0.19]
\node[anchor=south west,inner sep=0] (image) at (0.5,0)
  {\includegraphics[width=0.19\textwidth]{figures/DstNeutPTasymPi0MC}};
\end{tikzpicture}
\begin{tikzpicture}[scale=0.19]
\node[anchor=south west,inner sep=0] (image) at (0.5,0)
  {\includegraphics[width=0.19\textwidth]{figures/Dstar0PTPi0MC}};
\end{tikzpicture}
\begin{tikzpicture}[scale=0.19]
\node[anchor=south west,inner sep=0] (image) at (0.5,0)
  {\includegraphics[width=0.19\textwidth]{figures/DstNeutCLPi0Data}};
\begin{scope}[x={(image.south east)},y={(image.north west)}]
\node[] at (0.5, -0.1) {(a)};
\end{scope}
\end{tikzpicture}
\begin{tikzpicture}[scale=0.19]
\node[anchor=south west,inner sep=0] (image) at (0.5,0)
  {\includegraphics[width=0.19\textwidth]{figures/DstNeutIsNotEPi0Data}};
\begin{scope}[x={(image.south east)},y={(image.north west)}]
\node[] at (0.5, -0.1) {(b)};
\end{scope}
\end{tikzpicture}
\begin{tikzpicture}[scale=0.19]
\node[anchor=south west,inner sep=0] (image) at (0.5,0)
  {\includegraphics[width=0.19\textwidth]{figures/DstNeutPTtotPi0Data}};
\begin{scope}[x={(image.south east)},y={(image.north west)}]
\node[] at (0.5, -0.1) {(c)};
\end{scope}
\end{tikzpicture}
\begin{tikzpicture}[scale=0.19]
\node[anchor=south west,inner sep=0] (image) at (0.5,0)
  {\includegraphics[width=0.19\textwidth]{figures/DstNeutPTasymPi0Data}};
\begin{scope}[x={(image.south east)},y={(image.north west)}]
\node[] at (0.5, -0.1) {(d)};
\end{scope}
\end{tikzpicture}
\begin{tikzpicture}[scale=0.19]
\node[anchor=south west,inner sep=0] (image) at (0.5,0)
  {\includegraphics[width=0.19\textwidth]{figures/Dstar0PTPi0Data}};
\begin{scope}[x={(image.south east)},y={(image.north west)}]
\node[] at (0.5, -0.1) {(e)};
\end{scope}
\end{tikzpicture}
\caption{\small{Distributions of the input variables to the second stage BDT for
the $D^* \rightarrow D\pi^0$ decay mode. The top and bottom rows show plots for
simulation and data, respectively. The variables are labelled by letters: (a)
shows the product of the confident level, CL, of the two daughter photons, (b)
the product of the photon IsNotE, (c) the average photon $p_T$, (d) the photon
asymmetry and (e) the $D^*$ meson $p_T$. The red points show the distributions
before the cut on BDT2, and the blue points show the distributions after.}}
\label{fig:Pi0BDT2Variables}
\end{figure}

\afterpage{ \clearpage % Flush earlier floats (otherwise order might not be
\begin{landscape}
\begin{table}[]
\centering
\resizebox{.82\paperheight}{!}{ % with resizebox only specify one of {width}{height} as text is not scaled proportionally.
\begin{tabular}{lll}
\hline
Variable                    & Description
                            & BDT Importance ($\times 10^{-2}$)
                            \\ \hline
log$_{10}$(Bu\_RHO\_BPV)    & $\Delta (\rho )$ (cylindrical coordinates) between
the end vertex of the $B$ and the best primary vertex
                            & 9.01
                            \\
bach\_PT                    & Bachelor transverse momentum
                            & 7.52                              
                            \\
log$_{10}$(D0\_RHO\_BPV)    & $\Delta(\rho )$ (cylindrical coordinates) between 
the end vertex of the $D$ and the best primary vertex         
                            & 7.41
                            \\
log(1-D0\_DIRA\_BPV)        & DIRA is the cosine of the angle between the $D$
momentum and the vector from the primary to the secondary vertex 
                            & 6.96
                            \\
log(1-Bu\_DIRA\_BPV)        & DIRA is the cosine of the angle between the $B$
momentum and the vector from the primary to the secondary vertex     
                            & 6.96
                            \\
bach\_P                     & Bachelor momentum
                            & 6.59                              
                            \\
log(Bu\_FDCHI2\_OWNPV)      & Flight distance $\chi^2$ of the $B$ with respect to 
the primary vertex
                            & 6.32
                            \\
Bu\_ptasy\_1.50             & $P_{T}$ asymmetry of the $B$ candidate using a cone 
radius of 1.50
                            & 6.31                              
                            \\
log$_{10}$(Bu\_MIPCHI2\_PV) & Minimal $\chi^2$ for the impact parameter of the $B$ 
with respect to the primary vertex
                            & 5.89
                            \\
log$_{10}$(Bu\_FD\_BPV)     & Flight  distance of the $B$ with respect to the 
primary vertex
                            & 5.42
                            \\
log(D0\_FDCHI2\_OWNPV)      & Flight distance $\chi^2$ of the $D$ with respect to
the primary vertex          & 5.40
                            \\
log(D0\_MIPCHI2\_PV)        & Minimal $\chi^2$ for the impact parameter of the $D$
with respect to the primary vertex
                            & 5.34
                            \\
D0\_AMAXDOCA                & Distance of closest approach of the $D$ daughters
                            & 5.03
                            \\
Bu\_VTXCHI2DOF              & Vertex $\chi^2$ per degree of freedom of the $B$
                            & 4.96
                            \\
log$_{10}$(D0\_FD\_OWNPV)   & Flight distance of the $D$ with respect to the
primary vertex              & 3.76
                            \\
D0\_VTXCHI2DOF              & Vertex $\chi^2$ per degree of freedom of the $D$
                            & 3.73
                            \\
D0\_LT\_BPV                 & $D$ lifetime (laboratory frame) with respect to 
the primary vertex          & 3.41
                            \\ \hline
\end{tabular}
} \caption{\small{Variables used in the training of the first stage BDT, ranked
by importance. The name Bu refers to the $B$, bach to the $\pi/K$ from the
$B^{\pm}$ decay vertex, and D0 to the $D$.}}
\label{table:bdt1TrainingVar}
\end{table}

\begin{table}[]
\centering
\resizebox{.82\paperheight}{!}{
\begin{tabular}{lllll}
\hline
\multicolumn{2}{l}{Variable} &
Description &
\multicolumn{2}{l}{BDT Importance$(\times 10^{-2})$} \\
$D^*\rightarrow D\pi^0$ & $D^*\rightarrow D\gamma$ &
 &
$D^*\rightarrow D\pi^0$ & $D^*\rightarrow D\gamma$ \\ \hline
$D^{*}_{P_{T}}$ & $D^{*}_{P_{T}}$ & 
Transverse momentum of $D^*$ & 
2.117 & 2.828 \\
$\gamma^{1}_{CL} \times \gamma^{2}_{CL}$ & $\gamma_{CL}$ & 
Discriminates against non-EM deposits & 
2.098 & 2.572 \\
$\gamma^{1}_{IsNotE} \times \gamma^{2}_{IsNotE}$ & $\gamma_{IsNotE}$ & 
Discriminates against electrons & 
2.078 & 2.406 \\
- & $\gamma_{P_{T}}$ & 
Transverse energy of photon & 
- & 2.194 \\
$(\gamma^{1}_{P_{T}} + \gamma^{2}_{P_{T}})/2$ & - & 
Average transverse energy of photons & 
1.991 & - \\
$(\gamma^{1}_{P_{T}} - \gamma^{2}_{P_{T}})/( \gamma^{1}_{P_{T}} + \gamma^{2}_{P_{T}})$ & - & 
Asymmetry of transverse energy of photons 
& 1.716 & - \\ \hline
\end{tabular}
}
\caption{\small{Variables used in the training of the second stage BDT, ranked
by importance. $\gamma^{1}$ and $\gamma^{2}$ refer to the two daughter photons of the
$\pi^0$.}}
\label{table:bdt2TrainingVar}
\end{table}
\end{landscape}
}

\section{2D Mass Fit} \label{sec:massfit}

2D maximum likelihood fits to the $m[D^*h]$ and $\Delta_M$ mass spectrums are
performed to Cabibbo-favoured $B^{\pm}\rightarrow (D^*\rightarrow
[K^{\pm}\pi^{\mp}]_D\pi^0)h^{\pm}$ and $B^{\pm}\rightarrow (D^*\rightarrow
[K^{\pm}\pi^{\mp}]_D\gamma)h^{\pm}$ decays, in order to understand the
background compositions. Here, $h$ represents a pion or kaon, and these two
cases are fitted simultaneously. The fitting regions lies within the ranges 5050
$<$ $m[D^*h]$ $<$ 5800 MeV/$c^2$ and 50 (135) $<$ $\Delta_M$ $<$ 210 MeV/$c^2$
for the $D\gamma$ ($D\pi^0$) final states.

The physics backgrounds that fall within this region can be split into three
categories:
\begin{enumerate}
	\item Partially reconstructed backgrounds of the form $D^{*0}h+X$, where the
particle $X$ is missed. These components, $B^{\pm}\rightarrow
(D^{*0}\rightarrow [K^{\pm}\pi^{\mp}]_D \pi^0/\gamma)h^{*\pm}$, peak in the
$\Delta_M$ signal region, due to the presence of a true $D^{*0}$ meson, but sit
lower than signal in $m[D^*h]$ mass.
  \item Over-reconstructed backgrounds formed by the combindation of a fully
	reconstructed $D^0h$ candidate from $B^{\pm}\rightarrow
[K^{\pm}\pi^{\mp}]_Dh^{\pm}$ decays and a neutral particle in the event.
These backgrounds are flat in $\Delta_M$ phase space and sit higher than the
signal component in $m[D^*h]$ mass.
  \item Mis-reconstructed backgrounds, where a pion in the true decay has been missed and
	a $\pi^0$ or $\gamma$ added to the $D^0$ meson to form a $D^{*0}$ candidate.
$B^{\pm}\rightarrow [K^{\pm}\pi^{\mp}]_Dh^{*\pm}$ and $B^{0}\rightarrow
(D^{*\pm}\rightarrow [K^{\pm}\pi^{\mp}]_D \pi^{\pm})h^{\mp}$ decays peak
within the $m[D^*h]$ mass signal region, but are flat in $\Delta_M$ phase
space.
\end{enumerate}

There are also fit components within each neutral mode to account for
cross-feed from the alternative $D^*$ final state. 

Decays involving $B^{\pm}$ mesons have the potential to $CP$ violate. This
rendered the original 1D maximum likelihood fit to the $m[D^*h]$ mass spectrum
unsuitable, as decays of this type form part of the mis-reconstructed
background that sits under the the signal peaks. The 1D technique either suffered
from fit instabilities with shape parameters and asymmetries floating, or large
systematics with parameters fixed from simulation. Signal is the only component
that peaks in both $m[D^*h]$ and $\Delta_M$, and so the 2D method provides an
extra constraint on the signal yield to protect against $CP$ violating
backgrounds.

Background and signal simulation samples are processed using the same selection
procedure as descibed in Sec.~\ref{sec:selections}. The shape of these
post-selection samples are then parameterised by separate 1D PDFs in bins of
$\Delta_M$ and $m[D^*h]$. The signal components in both mass spectrums are
described by the sum of two Crystal Ball (CB) functions, which are asymmetric
Gaussian-like functions. One of the CB functions has a small radiative tail
extending to lower invariant mass, and the second has a tail extending to higher
invariant mass in order to describe the reconstruction and resolution effects of
neutral particles. In the $m[D^*h]$ mass spectrum, all other fit components are
also modelled by single or sums of CB functions. In $\Delta_M$ phase space,
crossfeed, over-reconstructed and mis-reconstructed backgrounds are modelled
using a PDF describing combinatoric backgrounds for $D^{*}\rightarrow D^{0}X$
decays in bins of $m[D^*]-m[D^0]$. The functional form of this shape, where $m_0$
represents the kinematic threshold, is:
\begin{equation}
f(m)=(1-e^{-\frac{m-m_0}{C}})\times(\frac{m}{m_0})^A+B\times(\frac{m}{m_0}-1)
\label{eq:DstD0BG}
\end{equation}
Partially reconstructed components are modelled by the sum of two CB functions
and a $D^*$ decaying to $D^0X$ background shape. The former describes the peak
in delta mass due to the presence of a true $D^{*0}$ meson. Mis-reconstructed
components are formed from the sum of the individual 2D PDFs describing
$B^{\pm}\rightarrow D^0\rho$ and
$B^{0}\rightarrow[D^0\pi^{\pm}]_{D^{*\pm}}\pi^{\mp}$ decays.

It should be noted that to obtain the PDFs of the signal components, Run1, 2015 and 2016 simulation samples are used. Currently, the background components are modelled using Run1 simulation only. 

When incorporating these PDFs into a 2D model, the $\Delta_M$ shape parameters
are fixed to the values obtained from the 1D fits. The tail parameters of the CB
functions in $m[D^*h]$ invariant mass are also fixed, but the means and widths
are constructed as polynomial functions of $\Delta_M$:  
\begin{equation}
x = a_0 + a_1 \Delta_M + a_2 \Delta_M^2
\label{eq:2dPolynomial}
\end{equation}
Here, $x$ represents the mean or width of a CB shape as a function of $m[D^*h]$.
The correlation between the two invariant mass spectrums is modelled in the fit
using these polynomial coefficients. Comparing the 2D plots of the data samples
shown in Fig.~\ref{fig:2dFitsPi0Run1} and Fig.~\ref{fig:2dFitsGammaRun1}, it can
be seen that these variables exhibit greater correlation in the $D\gamma$ final
state than in the $D\pi^0$ final state. This is due to the extra constraint on
the $\pi^0$ mass in calculating the re-fitted $m[D^*h]$ mass variable during
reconstruction, as described at the beginning of Sec.~\ref{sec:selections}. In
the 2D fits to simulation for the $D^*\rightarrow D\pi^0$ mode, a linear
polynomial is found to be sufficient to model this correlation, and so the $a_2$
parameter is set to zero.

The data, PDF and residuals in 2D for Run1 (Run1 and Run2) are shown in
Fig.~\ref{fig:2dFitsPi0Run1} (Fig.~\ref{fig:2dFitsPi0Total}) and
Fig.~\ref{fig:2dFitsGammaRun1} (Fig.~\ref{fig:2dFitsGammaTotal}), and the
corresponding projections on the $m[D^*h]$ and $\Delta_M$ axes in
Fig.~\ref{fig:ProjectionsPi0Run1} (Fig.~\ref{fig:ProjectionsPi0Total}) and
Fig.~\ref{fig:ProjectionsGammaRun1} (Fig.~\ref{fig:ProjectionsGammaTotal}), for
the $D\pi^0$ and $D\gamma$ final states, respectively. All PDF parameters are
fixed from simulation except the $a_0$ values of the widths describing the
mis-reconstructed components in $m[D^{*}h]$, which are allowed to vary over a
given range. The threshold parameters of the $D^{*}\rightarrow D^0X$
backgrounds in the $D\pi^0$ final state are also given the freedom to float.

\section{Results} \label{sec:results}

The signal yields taken from these fits are listed in Table~\ref{table:yields}.
It should be noted that this is using a preliminary optimisation, and the
working points for the BDTs and the mass window of the $\pi^0$ meson will be
re-optimised with respect to a more appropriate figure of merit, for example to
minimise the error on the physics observables.

\begin{table}[H]
\resizebox{\textwidth}{!}{%
\begin{tabular}{|c|c|c|c|c|}
\hline
\multirow{2}{*}{\textbf{Decay Mode}} & \multicolumn{2}{c|}{\textbf{Run1 Yield (3 fb$^{-1}$)}} & \multicolumn{2}{c|}{\textbf{Run1 and Run2 Yield (6.6 fb$^{-1}$)}} \\ \cline{2-5} 
 & \textbf{$B^{\pm}\rightarrow D^*\pi^{\pm}$} & \textbf{$B^{\pm}\rightarrow D^*K^{\pm}$} & \textbf{$B^{\pm}\rightarrow D^*\pi^{\pm}$} & \textbf{$B^{\pm}\rightarrow D^*K^{\pm}$} \\ \hline
$B^{\pm}\rightarrow (D^{*}\rightarrow [K^{\pm}\pi^{\mp}]_D \pi^0)\pi^{\pm}$ & 2372 $\pm$ 60 & 121 $\pm$ 13 & 8696 $\pm$ 112 & 478 $\pm$ 26 \\ \hline
$B^{\pm}\rightarrow (D^{*}\rightarrow [K^{\pm}\pi^{\mp}]_D \gamma)\pi^{\pm}$ & 10862 $\pm$ 132 & 604 $\pm$ 30 & 40102 $\pm$ 252 & 2125 $\pm$ 17 \\ \hline
\end{tabular}%
} \caption{\small{Signal yields and errors extracted from the 2D fit results for Run1, then Run1 and
Run2 data.}} \label{table:yields}
\end{table}

Focus will now be paid to the Run1 mass fit results, which are more reliable at
this stage due to the lack of Run2 background simulation samples.

% Yield extrapolation

The fit is set up to measure the physics observables $R^{K\pi , \pi^0
/\gamma}_{K/\pi}$, defined by Eq.~\ref{eq:R_DstK_Dstpi}. The results are give
below:
\begin{align*}
R^{K\pi , \pi^0}_{K/\pi} &= (8.003 \pm 0.886) \times 10^{-2} \\
R^{K\pi , \gamma}_{K/\pi} &= (8.689 \pm 0.447) \times 10^{-2} 
\end{align*} 
The errors given are statistical and taken straight from the fit result.
$R^{K\pi , \pi^0}_{K/\pi}$ and $R^{K\pi , \gamma}_{K/\pi}$ are consistent with
the PDG value of $R^{K\pi}_{K/\pi} = (8.10 \pm 0.15) \times 10^{-2}$ within
1$\sigma$ and 2$\sigma$, respectively. The partially reconstructed analysis of
these modes, ~\cite{PartReco}, gave the combined measurement ($R^{K\pi}_{K/\pi}
= 7.768 \pm 0.038$ (stat) $\pm 0.066$ (syst)) $\times 10^{-2}$, which these
results are also consistent with. 

In calculating these values for Run1 data, the percentage of
$B^{\pm}\rightarrow D^*K^{\pm}$ decays that pass the PID requirement on the
kaon is taken to be 64\%, from the studies of partially reconstructed decays.
This efficiency will be updated with a value calibrated to the data used in
this analysis, but is not expected to change significantly.


\begin{itemize}
\item Extrapolation to signal modes to give expected number in CP modes and ADS mode.
% \item Confident we understand the background - relative yields are in line with
% what we expect looking at simulation reconstruction and selection efficiencies.
% Compare to part reco study. 
% \item Fit stability problems with floating parameters - interellation of
% parameters in both dimensions and correlation coefficients. Still trying to
% understand with toy study.
% \item Mis-reco shape needs to be brought more under control.
\end{itemize}

\begin{figure}[H]
\centering
\begin{tikzpicture}[scale=0.32]
\node[anchor=south west,inner sep=0] (image) at (0.5,0)
  {\includegraphics[width=0.32\textwidth]{/Users/alexandra/Analysis/Bu2Dst0h_2d/FittingProgramme/build/data_Run1/pi0_pi_kpi_total_0_2dData.pdf}};
\end{tikzpicture}
\begin{tikzpicture}[scale=0.32]
\node[anchor=south west,inner sep=0] (image) at (0.5,0)
  {\includegraphics[width=0.32\textwidth]{/Users/alexandra/Analysis/Bu2Dst0h_2d/FittingProgramme/build/data_Run1/pi0_pi_kpi_total_0_2dPDF.pdf}};
\end{tikzpicture}
\begin{tikzpicture}[scale=0.32]
\node[anchor=south west,inner sep=0] (image) at (0.5,0)
  {\includegraphics[width=0.32\textwidth]{/Users/alexandra/Analysis/Bu2Dst0h_2d/FittingProgramme/build/data_Run1/pi0_pi_kpi_total_0_2dResiduals.pdf}};
\end{tikzpicture}
\begin{tikzpicture}[scale=0.32]
\node[anchor=south west,inner sep=0] (image) at (0.5,0)
  {\includegraphics[width=0.32\textwidth]{/Users/alexandra/Analysis/Bu2Dst0h_2d/FittingProgramme/build/data_Run1/pi0_k_kpi_total_0_2dData.pdf}};
\begin{scope}[x={(image.south east)},y={(image.north west)}]
\node[] at (0.5, -0.1) {(a)};
\end{scope}
\end{tikzpicture}
\begin{tikzpicture}[scale=0.32]
\node[anchor=south west,inner sep=0] (image) at (0.5,0)
  {\includegraphics[width=0.32\textwidth]{/Users/alexandra/Analysis/Bu2Dst0h_2d/FittingProgramme/build/data_Run1/pi0_k_kpi_total_0_2dPDF.pdf}};
\begin{scope}[x={(image.south east)},y={(image.north west)}]
\node[] at (0.5, -0.1) {(b)};
\end{scope}
\end{tikzpicture}
\begin{tikzpicture}[scale=0.32]
\node[anchor=south west,inner sep=0] (image) at (0.5,0)
  {\includegraphics[width=0.32\textwidth]{/Users/alexandra/Analysis/Bu2Dst0h_2d/FittingProgramme/build/data_Run1/pi0_k_kpi_total_0_2dResiduals.pdf}};
\begin{scope}[x={(image.south east)},y={(image.north west)}]
\node[] at (0.5, -0.1) {(c)};
\end{scope}
\end{tikzpicture}
\caption{\small{$\Delta_M$ vs. $m[D^*h]$ distributions of (a) Run1 data, (b)
2D PDFs, (c) 2D residuals for $B^{\pm}\rightarrow (D^*\rightarrow
[K^{\pm}\pi^{\mp}]_D\pi^0)\pi^{\pm}$ (top) and $B^{\pm}\rightarrow
(D^*\rightarrow [K^{\pm}\pi^{\mp}]_D\pi^0)K^{\pm}$ (bottom) decays.}}
\label{fig:2dFitsPi0Run1}
\end{figure}

\begin{figure}[H]
\centering
\begin{tikzpicture}[scale=0.48]
\node[anchor=south west,inner sep=0] (image) at (0.5,0)
  {\includegraphics[width=0.48\textwidth]{/Users/alexandra/Analysis/Bu2Dst0h_2d/FittingProgramme/build/data_Run1/pi0_pi_kpi_total_0_buMass.pdf}};
\end{tikzpicture}
\begin{tikzpicture}[scale=0.48]
\node[anchor=south west,inner sep=0] (image) at (0.5,0)
  {\includegraphics[width=0.48\textwidth]{/Users/alexandra/Analysis/Bu2Dst0h_2d/FittingProgramme/build/data_Run1/pi0_pi_kpi_total_0_deltaMass.pdf}};
\end{tikzpicture}
\begin{tikzpicture}[scale=0.48]
\node[anchor=south west,inner sep=0] (image) at (0.5,0)
  {\includegraphics[width=0.48\textwidth]{/Users/alexandra/Analysis/Bu2Dst0h_2d/FittingProgramme/build/data_Run1/pi0_k_kpi_total_0_buMass.pdf}};
\end{tikzpicture}
\begin{tikzpicture}[scale=0.48]
\node[anchor=south west,inner sep=0] (image) at (0.5,0)
  {\includegraphics[width=0.48\textwidth]{/Users/alexandra/Analysis/Bu2Dst0h_2d/FittingProgramme/build/data_Run1/pi0_k_kpi_total_0_deltaMass.pdf}};
\end{tikzpicture}
\caption{\small{1D projections onto the $m[D^*h]$ (left) and $\Delta_M$ (right)
axes for Run1 data, for $B^{\pm}\rightarrow (D^*\rightarrow
[K^{\pm}\pi^{\mp}]_D\pi^0)\pi^{\pm}$ (top) and $B^{\pm}\rightarrow
(D^*\rightarrow [K^{\pm}\pi^{\mp}]_D\pi^0)K^{\pm}$ (bottom) decays.}}
\label{fig:ProjectionsPi0Run1}
\end{figure}

\begin{figure}[H]
\centering
\begin{tikzpicture}[scale=0.32]
\node[anchor=south west,inner sep=0] (image) at (0.5,0)
  {\includegraphics[width=0.32\textwidth]{/Users/alexandra/Analysis/Bu2Dst0h_2d/FittingProgramme/build/data_Run1/gamma_pi_kpi_total_0_2dData.pdf}};
\end{tikzpicture}
\begin{tikzpicture}[scale=0.32]
\node[anchor=south west,inner sep=0] (image) at (0.5,0)
  {\includegraphics[width=0.32\textwidth]{/Users/alexandra/Analysis/Bu2Dst0h_2d/FittingProgramme/build/data_Run1/gamma_pi_kpi_total_0_2dPDF.pdf}};
\end{tikzpicture}
\begin{tikzpicture}[scale=0.32]
\node[anchor=south west,inner sep=0] (image) at (0.5,0)
  {\includegraphics[width=0.32\textwidth]{/Users/alexandra/Analysis/Bu2Dst0h_2d/FittingProgramme/build/data_Run1/gamma_pi_kpi_total_0_2dResiduals.pdf}};
\end{tikzpicture}
\begin{tikzpicture}[scale=0.32]
\node[anchor=south west,inner sep=0] (image) at (0.5,0)
  {\includegraphics[width=0.32\textwidth]{/Users/alexandra/Analysis/Bu2Dst0h_2d/FittingProgramme/build/data_Run1/gamma_k_kpi_total_0_2dData.pdf}};
\begin{scope}[x={(image.south east)},y={(image.north west)}]
\node[] at (0.5, -0.1) {(a)};
\end{scope}
\end{tikzpicture}
\begin{tikzpicture}[scale=0.32]
\node[anchor=south west,inner sep=0] (image) at (0.5,0)
  {\includegraphics[width=0.32\textwidth]{/Users/alexandra/Analysis/Bu2Dst0h_2d/FittingProgramme/build/data_Run1/gamma_k_kpi_total_0_2dPDF.pdf}};
\begin{scope}[x={(image.south east)},y={(image.north west)}]
\node[] at (0.5, -0.1) {(b)};
\end{scope}
\end{tikzpicture}
\begin{tikzpicture}[scale=0.32]
\node[anchor=south west,inner sep=0] (image) at (0.5,0)
  {\includegraphics[width=0.32\textwidth]{/Users/alexandra/Analysis/Bu2Dst0h_2d/FittingProgramme/build/data_Run1/gamma_k_kpi_total_0_2dResiduals.pdf}};
\begin{scope}[x={(image.south east)},y={(image.north west)}]
\node[] at (0.5, -0.1) {(c)};
\end{scope}
\end{tikzpicture}
\caption{\small{$\Delta_M$ vs. $m[D^*h]$ distributions of (a) Run1 data, (b)
2D PDFs, (c) 2D residuals for $B^{\pm}\rightarrow (D^*\rightarrow
[K^{\pm}\pi^{\mp}]_D\gamma)\pi^{\pm}$ (top) and $B^{\pm}\rightarrow
(D^*\rightarrow [K^{\pm}\pi^{\mp}]_D\gamma)K^{\pm}$ (bottom) decays.}}
\label{fig:2dFitsGammaRun1}
\end{figure}

\begin{figure}[H]
\centering
\begin{tikzpicture}[scale=0.48]
\node[anchor=south west,inner sep=0] (image) at (0.5,0)
  {\includegraphics[width=0.48\textwidth]{/Users/alexandra/Analysis/Bu2Dst0h_2d/FittingProgramme/build/data_Run1/gamma_pi_kpi_total_0_buMass.pdf}};
\end{tikzpicture}
\begin{tikzpicture}[scale=0.48]
\node[anchor=south west,inner sep=0] (image) at (0.5,0)
  {\includegraphics[width=0.48\textwidth]{/Users/alexandra/Analysis/Bu2Dst0h_2d/FittingProgramme/build/data_Run1/gamma_pi_kpi_total_0_deltaMass.pdf}};
\end{tikzpicture}
\begin{tikzpicture}[scale=0.48]
\node[anchor=south west,inner sep=0] (image) at (0.5,0)
  {\includegraphics[width=0.48\textwidth]{/Users/alexandra/Analysis/Bu2Dst0h_2d/FittingProgramme/build/data_Run1/gamma_k_kpi_total_0_buMass.pdf}};
\end{tikzpicture}
\begin{tikzpicture}[scale=0.48]
\node[anchor=south west,inner sep=0] (image) at (0.5,0)
  {\includegraphics[width=0.48\textwidth]{/Users/alexandra/Analysis/Bu2Dst0h_2d/FittingProgramme/build/data_Run1/gamma_k_kpi_total_0_deltaMass.pdf}};
\end{tikzpicture}
\caption{\small{1D projections onto the $m[D^*h]$ (left) and $\Delta_M$ (right)
axes for Run1 data, for $B^{\pm}\rightarrow (D^*\rightarrow
[K^{\pm}\pi^{\mp}]_D\gamma)\pi^{\pm}$ (top) and $B^{\pm}\rightarrow
(D^*\rightarrow [K^{\pm}\pi^{\mp}]_D\gamma)K^{\pm}$ (bottom) decays.}}
\label{fig:ProjectionsGammaRun1}
\end{figure}

\begin{figure}[H]
\centering
\begin{tikzpicture}[scale=0.32]
\node[anchor=south west,inner sep=0] (image) at (0.5,0)
  {\includegraphics[width=0.32\textwidth]{/Users/alexandra/Analysis/Bu2Dst0h_2d/FittingProgramme/build/all_data/pi0_pi_kpi_total_0_2dData.pdf}};
\end{tikzpicture}
\begin{tikzpicture}[scale=0.32]
\node[anchor=south west,inner sep=0] (image) at (0.5,0)
  {\includegraphics[width=0.32\textwidth]{/Users/alexandra/Analysis/Bu2Dst0h_2d/FittingProgramme/build/all_data/pi0_pi_kpi_total_0_2dPDF.pdf}};
\end{tikzpicture}
\begin{tikzpicture}[scale=0.32]
\node[anchor=south west,inner sep=0] (image) at (0.5,0)
  {\includegraphics[width=0.32\textwidth]{/Users/alexandra/Analysis/Bu2Dst0h_2d/FittingProgramme/build/all_data/pi0_pi_kpi_total_0_2dResiduals.pdf}};
\end{tikzpicture}
\begin{tikzpicture}[scale=0.32]
\node[anchor=south west,inner sep=0] (image) at (0.5,0)
  {\includegraphics[width=0.32\textwidth]{/Users/alexandra/Analysis/Bu2Dst0h_2d/FittingProgramme/build/all_data/pi0_k_kpi_total_0_2dData.pdf}};
\begin{scope}[x={(image.south east)},y={(image.north west)}]
\node[] at (0.5, -0.1) {(a)};
\end{scope}
\end{tikzpicture}
\begin{tikzpicture}[scale=0.32]
\node[anchor=south west,inner sep=0] (image) at (0.5,0)
  {\includegraphics[width=0.32\textwidth]{/Users/alexandra/Analysis/Bu2Dst0h_2d/FittingProgramme/build/all_data/pi0_k_kpi_total_0_2dPDF.pdf}};
\begin{scope}[x={(image.south east)},y={(image.north west)}]
\node[] at (0.5, -0.1) {(b)};
\end{scope}
\end{tikzpicture}
\begin{tikzpicture}[scale=0.32]
\node[anchor=south west,inner sep=0] (image) at (0.5,0)
  {\includegraphics[width=0.32\textwidth]{/Users/alexandra/Analysis/Bu2Dst0h_2d/FittingProgramme/build/all_data/pi0_k_kpi_total_0_2dResiduals.pdf}};
\begin{scope}[x={(image.south east)},y={(image.north west)}]
\node[] at (0.5, -0.1) {(c)};
\end{scope}
\end{tikzpicture}
\caption{\small{$\Delta_M$ vs. $m[D^*h]$
distributions of (a) Run1, Run2 data, (b) 2D PDFs, (c) 2D residuals for
$B^{\pm}\rightarrow (D^*\rightarrow [K^{\pm}\pi^{\mp}]_D\pi^0)\pi^{\pm}$ (top)
and $B^{\pm}\rightarrow (D^*\rightarrow [K^{\pm}\pi^{\mp}]_D\pi^0)K^{\pm}$
(bottom) decays.}}
\label{fig:2dFitsPi0Total}
\end{figure}

\begin{figure}[H]
\centering
\begin{tikzpicture}[scale=0.48]
\node[anchor=south west,inner sep=0] (image) at (0.5,0)
  {\includegraphics[width=0.48\textwidth]{/Users/alexandra/Analysis/Bu2Dst0h_2d/FittingProgramme/build/all_data/pi0_pi_kpi_total_0_buMass.pdf}};
\end{tikzpicture}
\begin{tikzpicture}[scale=0.48]
\node[anchor=south west,inner sep=0] (image) at (0.5,0)
  {\includegraphics[width=0.48\textwidth]{/Users/alexandra/Analysis/Bu2Dst0h_2d/FittingProgramme/build/all_data/pi0_pi_kpi_total_0_deltaMass.pdf}};
\end{tikzpicture}
\begin{tikzpicture}[scale=0.48]
\node[anchor=south west,inner sep=0] (image) at (0.5,0)
  {\includegraphics[width=0.48\textwidth]{/Users/alexandra/Analysis/Bu2Dst0h_2d/FittingProgramme/build/all_data/pi0_k_kpi_total_0_buMass.pdf}};
\end{tikzpicture}
\begin{tikzpicture}[scale=0.48]
\node[anchor=south west,inner sep=0] (image) at (0.5,0)
  {\includegraphics[width=0.48\textwidth]{/Users/alexandra/Analysis/Bu2Dst0h_2d/FittingProgramme/build/all_data/pi0_k_kpi_total_0_deltaMass.pdf}};
\end{tikzpicture}
\caption{\small{1D projections onto the $m[D^*h]$ (left) and $\Delta_M$ (right)
axes for Run1, Run2 data, for $B^{\pm}\rightarrow (D^*\rightarrow
[K^{\pm}\pi^{\mp}]_D\pi^0)\pi^{\pm}$ (top) and $B^{\pm}\rightarrow
(D^*\rightarrow [K^{\pm}\pi^{\mp}]_D\pi^0)K^{\pm}$ (bottom) decays.}}
\label{fig:ProjectionsPi0Total}
\end{figure}

\begin{figure}[H]
\centering
\begin{tikzpicture}[scale=0.32]
\node[anchor=south west,inner sep=0] (image) at (0.5,0)
  {\includegraphics[width=0.32\textwidth]{/Users/alexandra/Analysis/Bu2Dst0h_2d/FittingProgramme/build/all_data/gamma_pi_kpi_total_0_2dData.pdf}};
\end{tikzpicture}
\begin{tikzpicture}[scale=0.32]
\node[anchor=south west,inner sep=0] (image) at (0.5,0)
  {\includegraphics[width=0.32\textwidth]{/Users/alexandra/Analysis/Bu2Dst0h_2d/FittingProgramme/build/all_data/gamma_pi_kpi_total_0_2dPDF.pdf}};
\end{tikzpicture}
\begin{tikzpicture}[scale=0.32]
\node[anchor=south west,inner sep=0] (image) at (0.5,0)
  {\includegraphics[width=0.32\textwidth]{/Users/alexandra/Analysis/Bu2Dst0h_2d/FittingProgramme/build/all_data/gamma_pi_kpi_total_0_2dResiduals.pdf}};
\end{tikzpicture}
\begin{tikzpicture}[scale=0.32]
\node[anchor=south west,inner sep=0] (image) at (0.5,0)
  {\includegraphics[width=0.32\textwidth]{/Users/alexandra/Analysis/Bu2Dst0h_2d/FittingProgramme/build/all_data/gamma_k_kpi_total_0_2dData.pdf}};
\begin{scope}[x={(image.south east)},y={(image.north west)}]
\node[] at (0.5, -0.1) {(a)};
\end{scope}
\end{tikzpicture}
\begin{tikzpicture}[scale=0.32]
\node[anchor=south west,inner sep=0] (image) at (0.5,0)
  {\includegraphics[width=0.32\textwidth]{/Users/alexandra/Analysis/Bu2Dst0h_2d/FittingProgramme/build/all_data/gamma_k_kpi_total_0_2dPDF.pdf}};
\begin{scope}[x={(image.south east)},y={(image.north west)}]
\node[] at (0.5, -0.1) {(b)};
\end{scope}
\end{tikzpicture}
\begin{tikzpicture}[scale=0.32]
\node[anchor=south west,inner sep=0] (image) at (0.5,0)
  {\includegraphics[width=0.32\textwidth]{/Users/alexandra/Analysis/Bu2Dst0h_2d/FittingProgramme/build/all_data/gamma_k_kpi_total_0_2dResiduals.pdf}};
\begin{scope}[x={(image.south east)},y={(image.north west)}]
\node[] at (0.5, -0.1) {(c)};
\end{scope}
\end{tikzpicture}
\caption{\small{$\Delta_M$ vs. $m[D^*h]$
distributions of (a) Run1, Run2 data, (b) 2D PDFs, (c) 2D residuals for
$B^{\pm}\rightarrow (D^*\rightarrow [K^{\pm}\pi^{\mp}]_D\gamma)\pi^{\pm}$ (top)
and $B^{\pm}\rightarrow (D^*\rightarrow [K^{\pm}\pi^{\mp}]_D\gamma)K^{\pm}$
(bottom) decays.}}
\label{fig:2dFitsGammaTotal}
\end{figure}

\begin{figure}[H]
\centering
\begin{tikzpicture}[scale=0.48]
\node[anchor=south west,inner sep=0] (image) at (0.5,0)
  {\includegraphics[width=0.48\textwidth]{/Users/alexandra/Analysis/Bu2Dst0h_2d/FittingProgramme/build/all_data/gamma_pi_kpi_total_0_buMass.pdf}};
\end{tikzpicture}
\begin{tikzpicture}[scale=0.48]
\node[anchor=south west,inner sep=0] (image) at (0.5,0)
  {\includegraphics[width=0.48\textwidth]{/Users/alexandra/Analysis/Bu2Dst0h_2d/FittingProgramme/build/all_data/gamma_pi_kpi_total_0_deltaMass.pdf}};
\end{tikzpicture}
\begin{tikzpicture}[scale=0.48]
\node[anchor=south west,inner sep=0] (image) at (0.5,0)
  {\includegraphics[width=0.48\textwidth]{/Users/alexandra/Analysis/Bu2Dst0h_2d/FittingProgramme/build/all_data/gamma_k_kpi_total_0_buMass.pdf}};
\end{tikzpicture}
\begin{tikzpicture}[scale=0.48]
\node[anchor=south west,inner sep=0] (image) at (0.5,0)
  {\includegraphics[width=0.48\textwidth]{/Users/alexandra/Analysis/Bu2Dst0h_2d/FittingProgramme/build/all_data/gamma_k_kpi_total_0_deltaMass.pdf}};
\end{tikzpicture}
\caption{\small{1D projections onto the $m[D^*h]$ (left) and $\Delta_M$ (right)
axes for Run1, Run2 data, for $B^{\pm}\rightarrow (D^*\rightarrow
[K^{\pm}\pi^{\mp}]_D\gamma)\pi^{\pm}$ (top) and $B^{\pm}\rightarrow
(D^*\rightarrow [K^{\pm}\pi^{\mp}]_D\gamma)K^{\pm}$ (bottom) decays.}}
\label{fig:ProjectionsGammaTotal}
\end{figure}

\section{Thesis Outline} \label{sec:thesis}
Proposed thesis title:

\noindent Study of $B^{\pm}\rightarrow D^{*}K^{\pm}$ decays using the full LHCb dataset.

\noindent Suggested thesis outline including chapter and section headings:
\begin{enumerate}
\item Introduction
\item Theoretical Background \\
			Section headings: The Standard Model, $CP$ violation in $B$ decays (CKM
matrix, direct $CP$ violation, in mixing, in mixing and decay), Introduction to
$\phi_3$, Derivation of Physics Observables (extended ADS and GLW methods)
\item The LHCb Experiment \\
			Section headings: The LHC, The LHCb Detector (VELO, Tracking, RICH,
Calorimeters, Muon Systems, Trigger, Software), Neutral Particle Reconstruction 
\item Selection and 2D mass fit of $B^{\pm}\rightarrow (D^*\rightarrow [K^{\pm}\pi^{\mp}]_D\pi^0/\gamma)K^{\pm}$ events \\
			Section headings: Pre-selection, Charged Particle Selection (stage 1
multivariate analysis), Neutral Selection (stage 2 multivariate analysis),
Backgrounds, 2D Mass Fit \\
			Status: 2D mass fit under development, plan to
conduct toy studies within the next month. The selection and mass fit will also
be updated with 2018 data by Easter. 
\item $CP$ fit for $B^{\pm}\rightarrow (D^*\rightarrow [h^{\pm}h^{\mp}]_D\pi^0/\gamma)K^{\pm}$ decays \\
			Section headings: Fit Setup, Asymmetry and Efficiency Corrections,
Optimisation, Fit Results, Systematics \\
		Status: Fit extension to GLW and ADS modes still needs to be done, aim to achieve this by Easter 2018. Plan to evaluate the corrections, optimisation and systematics by summer 2018.
\item Extraction of Physics Parameters
\item Conclusions
\end{enumerate}

\printbibliography[heading=bibintoc,{title=References}]
\end{document}
