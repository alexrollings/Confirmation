\documentclass[oneside,12pt]{article} 

\usepackage[margin=2.5cm]{geometry} 
\usepackage{sidecap}
\usepackage{fullpage}
\geometry{a4paper}
\usepackage[final]{graphicx}
\usepackage{url}
\usepackage{amsmath, amssymb}
\usepackage{amsfonts}
\usepackage[mathscr]{euscript}
\usepackage{mathtools}
\usepackage{bbold}
\usepackage{float}
\usepackage{wrapfig}
\usepackage{sidecap}
\usepackage{caption}
\usepackage{subcaption}
\usepackage{multirow}
\usepackage{sidecap}
\usepackage{fancyhdr}
\usepackage{verbatim}
\usepackage[rflt]{floatflt}
\usepackage{titlesec}
\usepackage{gensymb}
\usepackage{enumerate}
\usepackage{cleveref} % allows referencing multiple figures in one go
\usepackage{pdflscape} % to allow table to be displayed in landscape mode
\usepackage{afterpage} % to allow table to be given entire page to itself
\usepackage{accents}
\usepackage{tikz} % drawing on images
\usepackage[sorting=none]{biblatex}
% \usepackage[backend=biber, bibencoding=utf8, style=authoryear, citestyle=authoryear]{biblatex}
\bibliography{Bibliography}
% \usepackage[square, comma, sort&compress]{natbib}
% \usepackage{bibentry}
\usepackage{lineno}

\linenumbers

\titleformat{\section}
  {\normalfont\fontsize{12}{15}\bfseries}{\thesection}{1em}{}

% \graphicspath{{/Users/alexandrarollings/Analysis/Confirmation/latex/figures/}}

% \renewcommand{\refname}{References}
\newcommand{\overbar}[1]{\mkern 1.5mu\overline{\mkern-1.5mu#1\mkern-1.5mu}\mkern
1.5mu}
\newcommand\brabar{\scalebox{.3}{(}\raisebox{-1.7pt}{$-$}\scalebox{.3}{)}}
\newcommand\brobor{\smash[b]{\raisebox{0.6\height}{\scalebox{0.5}{\tiny(}}{\mkern-1.5mu\scriptstyle-\mkern-1.5mu}\raisebox{0.6\height}{\scalebox{0.5}{\tiny)}}}}

\newcommand{\HRule}[1]{\rule{\linewidth}{#1}}     % Horizontal rule
% Remove 'In:' in Bibliography
\renewbibmacro{in:}{}
% Remove 'pp' in Bibliography
\DeclareFieldFormat{pages}{#1}


\makeatletter                            % Title
\def\printtitle{%                       
  {\centering \@title\par}}
\makeatother                                    

\makeatletter                            % Author
\def\printauthor{%                    
  {\centering \large \@author}}                
\makeatother                

\title{ \LARGE \textbf{Study of $B^{\pm} \rightarrow D^{*}K^{\pm}$ decays in the
interest of obtaining a direct measurement of $\phi_3$} \\ }

\author{
		Alexandra Rollings\\	
		Supervisor: Dr M. John\\	
}

\begin{document}
\begin{titlepage}
\begin{nolinenumbers}
\thispagestyle{empty} 

\begin{minipage}[c]{.15\linewidth}
  \includegraphics[width=\linewidth]{figures/Oxford}
\end{minipage}\hfill
\begin{minipage}[c]{.75\linewidth}
  \begin{flushright}
  \normalsize {Confirmation Report} 	% Subtitle
  \\ \normalsize \today			% Todays date
  \end{flushright}
\end{minipage}



\HRule{0.5pt} \\						% Upper rule
[2.0cm]
\printtitle 
\vspace{75pt}
\printauthor
\vfill
\begin{abstract}
\noindent
\\
  % The decay $B^{\pm} \rightarrow D^{*}\pi^{\pm}$, where $D^{*} \rightarrow
  % [K^{\pm}\pi^{\pm}]_D$ $\pi^{0}\text{/}\gamma$, is studied using 3.3
  % fb$^{-1}$ integrated luminosity in $pp$ collisions taken at the LHCb
  % experiment. This combines 2011, 2012 and 2015 data taken at center-of-mass
  % energies of 7 TeV, 8 TeV and 13 TeV, respectively. The techniques for
  % discriminating against combinatorial backgrounds are described and the
  % physics backgrounds are identified. Using a preliminary optimisation, the
  % number of events is reported and an extrapolation of event yields to the
  % modes sensitive to $\phi_3$ is made. 
\end{abstract}
\vfill
\end{nolinenumbers}
\end{titlepage}
\setcounter{page}{1}
\section{\normalsize Introduction}
% The Standard Model (SM) of particle physics is a quantum field theory which
% describes the electromagnetic, strong and weak interactions, and classifies
% all known elementary particles. Work over the last century has developed this
% successful theory, and many aspects of it have been validated by particle
% physics experiments around the world, a substantial contribution of which has
% been made by the CERN experiment in Geneva. However, the SM does fall short in
% a number of aspects, one of which being its failure to describe the
% matter-antimatter asymmetry we see in the observable Universe.  Quantities of
% the observable Universe is almost entirely matter
% dominated~\cite{UniAsymmetry}. It is therefore postulated that baryogenesis,
% the process that produced baryon asymmetry, occurred in the early Universe.
% Three Sakharaov conditions are required for this to be possible: I) baryon
% number violation, II) C and $CP$ violation, III) deviation from thermal
% equilibrium~\cite{Sakharov}.
The Standard Model (SM) of particle physics, the quantum field theory describing
three of the four fundamental forces, allows for $CP$ violating processes, but the
level at which these processes occur would need to be several orders of
magnitude larger to account for the Universal matter-antimatter asymmetry. This
suggest that new physics, beyond the SM, is needed to fully describe this
asymmetry. The LHCb experiment at CERN has been designed to collect
unprecedented samples of decays of $B$ mesons and study many $CP$-violating
processes of the SM. This report will describe an ongoing measurement of a SM
parameter critical to this endeavour. \\ 
\\
DESCRIPTION OF REPORT STRUCTURE.
\section{\normalsize Theoretical Background}
In the SM, the charged-current, weak interactions of quarks are described by the
SM Lagrangian term given in Eq.~\eqref{eq:CCLagrangian}.
\begin{equation}
  L_{CC}=-\frac{g}{\sqrt{2}}\overbar{q}_{Li}\gamma^{\mu}W_{\mu}^{+}(V_{CKM})_{ij}q_{Lj}
  + h.c.  
  \label{eq:CCLagrangian}
\end{equation}
\noindent Here, $W_{\mu}^{+}$ is the W-boson field, which couples to the
left-handed quark triplets, $\overbar{q}_{L}^{i}$ and $q_{L}^{j}$, where
$i,j=1,2,3$ are the generation numbers, and $V_{CKM}$ are elements of the
$3\times 3$ Cabbibo-Kobayashi-Maskawa (CKM) matrix~\cite{CKMTheory}. These
elements are coupling constants quantifying the strength of iter- and
intra-generational mixing, and can be represented by complex numbers with arbitrary phases.

For a $3\times 3$ unitary matrix, $3^2=9$ real parameters must be specified. Our
ability to absorb one phase into each quark field, but inability to observe an
overall common phase, removes $5$ of these parameters. $4$ degrees of freedom
therefore remain: $3$ rotations and $1$ complex phase. It is the presence of
this single irreducible phase that can generate $CP$ violation in this sector.

Verifying the unitarity of the CKM matrix is an important test of the
SM~\cite{CKMTheory}. This requirement can be summarised by
$\sum_{k=1}^{3}V_{ik}V^*_{jk}=\delta_{ij}$. Of particular interest is the
condition $V_{ud}V^{*}_{ub}+V_{cd}V^{*}_{cb}+V_{td}V^{*}_{tb}=0$, which forms
the \emph{Unitarity Triangle} (UT) in the complex plane. The geometry of the UT
allows all lengths and angles to be accessed experimentally, and thus it can be
over-constrained, and used as a probe of new physics in precision measurements
of the quark-mixing parameters. 

The angles in this triangle, $\phi_1$, $\phi_2$ and $\phi_3$, can be accessed
experimentally via the study of $B$ meson decays. For many years, the least well
known angle was $\phi_3=\arg(-V_{ud}V^{*}_{ub}/V_{cd}V^{*}_{cb})$ (more commonly
referred to as $\gamma$), but recently its precision now rivals that of
$\phi_2$. $\phi_3$ can be thought of as the Standard Candle of the UT, as it can
be accessed via tree-level decays of $B$ mesons with very low theoretical
uncertainty, $\frac{\delta\gamma}{\gamma}\sim 10^{-7}$. The current World
average value using a combination of tree-level measurements is
$\phi_3=(74.0^{+5.0}_{-5.8})$\degree~\cite{LatestGamma}.

It is also possible to constrain $\phi_3$ indirectly, by excluding tree-level
measurements and performing global fits to the CKM triangle, as is shown in
Fig.~\ref{fig:CKMIndirectFit}. The current best indirect measurement is
$\phi_3=(65.64^{+0.97}_{-3.42})$\degree~\cite{website:CKMFitter}. The
difference between the values of the direct and indirect measurements is of
great interest in the current climate. Over the past few years, flavour
anomalies have appeared in measurements of $\mathcal{B} (B^{0}_{s}\rightarrow
\phi \mu \mu)$~\cite{B2phimumu}, $\mathcal{B} (B^{0}\rightarrow K^{*0} \mu
\mu)$~\cite{B2Kstmumu}, $R_{K}$~\cite{Rk}, $R_{K^{*0}}$~\cite{Rkst} and
$R_{D^{(*)-}}$~\cite{RDst}. Though none have reached a significance level to
justify a discovery, some combinations have ???. These anomalies, and the
contention in the measurements of $\phi_3$, could be explained by the presence
of new, heavy particles. It is therefore motivating to reduce the uncertainty
on $\phi_{3}$ using tree-level measurements to a size comparable to that
extracted from the indirect measurements.
\begin{figure}[H]
	\centering \includegraphics[width=1\textwidth]{figures/CKMIndirectFit.eps}
\caption{\small{Diagram to show current measurements of the Unitarity Triangle
using only constraints from `loop' quantities in the ($\rho$-bar, $\eta$-bar)
plane. The black line represents the fit obtained by CKM fitter.}}
\label{fig:CKMIndirectFit} \vspace{-10pt}
\end{figure}
A significant contribution to this effort comes from studying tree-level
processes of the type $B^{\pm}\rightarrow D^{0(*)}K^{\pm (*)}$~\cite{B2DKD2hh,
DalitzRun1, DalitzRun2, B2DKstD2hh, B2DstKD2hh}. The ADS~\cite{ADSRef} and
GLW~\cite{GLWRef} methods can be employed to access $\phi_3$, the relative weak
phase between the $b\rightarrow c\overbar{u}s$ and $b\rightarrow u\overbar{c}s$
transitions, as seen in the Feynman diagrams in Fig.~\ref{fig:B2DstarKstar}.
This report will study one variant, $B^{\pm}\rightarrow D^{*}K^{\pm}$ decays,
where the neutral D$^{*}$ meson represents $D^{*0}$ or $\overbar{D}^{*0}$ and
is reconstructed as either $D\pi^{0}$ or $D\gamma$. 
% \begin{figure}[H]
% 	\centering
% 	\includegraphics[width=0.8\textwidth]{figures/B2DstarKstar}
% 	\caption{\small{Feynman diagrams for ${B}^{-}\rightarrow{D}^{(*)}{K}^{(*)-}$}.
% The decay to $\overbar{{D}}^{(*)0}$ is CKM
% $|V_{ub}V^{*}_{cs}/V_{cb}V^{*}_{us}|\approx0.4$ and colour ${F}_{CS}\approx1/3$
% suppressed with respect to (w.r.t.) the ${D}^{(*)0}$ final state. The
% interference of these two diagrams is the common mechanism for accessing
% $\phi_3$.}
% 	\label{fig:B2DstarKstar}
% 	\vspace{-10pt}
% \end{figure}

\begin{figure}[H]
\centering
\begin{tikzpicture}[scale=0.8]
\node[anchor=south west,inner sep=0] (image) at (0.5,0)
  {\includegraphics[width=0.8\textwidth]{figures/B2DstarKstar}};
\begin{scope}[x={(image.south east)},y={(image.north west)}]
\node[] at (0.21,1) {(a)};
\node[] at (0.71,1) {(b)};
\end{scope}
\end{tikzpicture}
	\caption{\small{Feynman diagrams for ${B}^{-}\rightarrow{D}^{(*)}{K}^{(*)-}$}.
The decay to $\overbar{{D}}^{(*)0}$ is CKM
$|V_{ub}V^{*}_{cs}/V_{cb}V^{*}_{us}|\approx0.4$ and colour ${F}_{CS}\approx1/3$
suppressed with respect to (w.r.t.) the ${D}^{(*)0}$ final state. The
interference of these two diagrams is the common mechanism for accessing
$\phi_3$.}
	\label{fig:B2DstarKstar}
	\vspace{-10pt}
\end{figure}

\noindent The amplitude for producing a neutral ${D}^*$ meson in the decay $B^-\rightarrow \tilde{D^*}K^-$ can be expressed as:
  \begin{equation}
    \tilde{D}^*=D^{*0}+r_Be^{i(\delta_B-\phi_3)}\overbar{D}^{*0}
    \label{eq:DstTilde}
  \end{equation}
\noindent Writing this in terms of odd and even $CP$ eigenstates, where
$D^{*}_{+}=\frac{D^{*0}+\overbar{D}^{*0}}{\sqrt{2}}$ and
$D^{*}_{-}=\frac{D^{*0}-\overbar{D}^{*0}}{\sqrt{2}}$:
  \begin{equation}
    \tilde{D}^*=\frac{D^{*}_{+}+D^{*}_{-}}{\sqrt{2}}+r_Be^{i(\delta_B-\phi_3)}\frac{D^{*}_{+}-D^{*}_{-}}{\sqrt{2}}
    \label{eq:DstTildeCP}
  \end{equation}
\noindent Here, $r_B=\frac{|A(B^{-}\rightarrow
\overbar{D}^{*0}K^-)|}{|A(B^{-}\rightarrow {D}^{*0}K^-)|}$ and $\delta_{B}$ is
the relative strong phase between these $B^-$ decays.
 
The $CP$ eigenvalue of the $D^*$ state is given by the product
$\lambda_{D^*}=\lambda_D\times \lambda_{\pi^0\text{/}\gamma} \times(-1)^l$. In
the case of the strong $D^*$ decay via $\pi^0$ emission, $\lambda_{\pi^0}=-1$
and $l=1$ therefore $\lambda_{D^*}=\lambda_D$ and $D^*_{\pm}\rightarrow
D_{\pm}\pi^0$. Where as, for $\gamma$ emission, $\lambda_{\gamma}=+1$ and
$l=1$, to conserve parity, therefore $\lambda_{D^*}=-\lambda_D$ and
$D^*_{\pm}\rightarrow D_{\mp}\gamma$.  This introduces a phase shift of $\pi$
($e^{i\pi}=-1$) between the two neutral $D$ mesons produced in the decay
$B^-\rightarrow \tilde{D}^*K^-$ in the $D\gamma$ mode. This logic was first
shown in~\cite{ADSDstar}.
  \begin{equation}
		\tilde{D^*}\rightarrow \tilde{D}\pi^0 :
\tilde{D}=D^0+r_Be^{i(\delta_B-\phi_3)}\overbar{D}^0 \label{eq:DTildePi0}
  \end{equation}
  \begin{equation}
		\tilde{D^*}\rightarrow \tilde{D}\gamma :
\tilde{D}=D^0+r_Be^{i(\delta_B+\pi-\phi_3)}\overbar{D}^0 \label{eq:DTildeGamma}
  \end{equation}
\begin{figure}[H]
  \centering
	\begin{tikzpicture}[scale=1.0]
	\node[anchor=south west,inner sep=0] (image) at (0.5,0)
		{\includegraphics[width=1.0\textwidth]{figures/B2DstKDiagram.png}};
	\begin{scope}[x={(image.south east)},y={(image.north west)}]
	% \node[] at (0.25,0.5) {(a)};
	% \node[] at (0.75,0.5) {(b)};
	\end{scope}
	\end{tikzpicture}
  \caption{\small{Decay diagram depicting ${B}^{-}\rightarrow{D}^{*}{K}^{-}$ to
  a general $D^{*}$ final state, $f(D^{*})$, and the complex conjugate decay.
  These decays proceed via 2 interfering amplitudes, $D^{*}$ or 
  $\overbar{D}^{*}$ states.}}
\label{fig:B2DstKDiagram} \vspace{-10pt}
\end{figure}
Labelling the amplitude of the diagram displayed in Fig.~\ref{fig:B2DstarKstar} (a)
as $A_{B}=A(B^{-}\rightarrow
D^{*0}K^{-})=A(B^{+}\rightarrow \overbar{D}^{*0}K^{+})$, as shown in Fig.~\ref{fig:B2DstKDiagram}, we can construct 4
equations for the decay amplitudes of $B^{-}$ mesons to final states
$f(\accentset{\brobor}{D}\phantom{})$:
\begin{equation}
    A_{Bf}^{\pi^{0}}=A(B^{-}\rightarrow
    [f(D)\pi^{0}]_{D^{*}}K^{-})=A_{B}A_{\pi^{0}}(A_{D}+\overbar{A}_{D}e^{-i\delta_D}r_{B}e^{(i(\delta_{B}-\phi_{3}))})
    \label{eq:A(Bf)pi0}
  \end{equation}
  \begin{equation}
		A_{B\overbar{f}}^{\pi^{0}}=A(B^{-}\rightarrow
[f(\overbar{D})\pi^{0}]_{D^{*}}K^{-})=A_{B}A_{\pi^{0}}(\overbar{A}e^{-i\delta_D}_{D}+A_{D}r_{B}e^{(i(\delta_{B}-\phi_{3}))})
\label{eq:A(Bfbar)pi0}
  \end{equation}
  \begin{equation}
    A_{Bf}^{\gamma}=A(B^{-}\rightarrow
    [f(D)\gamma]_{D^{*}}K^{-})=A_{B}A_{\gamma}(A_{D}+\overbar{A}_{D}e^{-i\delta_D}r_{B}e^{(i(\delta_{B}+\pi-\phi_{3}))})
    \label{eq:A(Bf)gamma}
  \end{equation}
  \begin{equation}
		A_{B\overbar{f}}^{\gamma}=A(B^{-}\rightarrow
[f(\overbar{D})\gamma]_{D^{*}}K^{-})=A_{B}A_{\gamma}(\overbar{A}_{D}e^{-i\delta_D}+A_{D}r_{B}e^{(i(\delta_{B}+\pi-\phi_{3}))})
\label{eq:A(Bfbar)gamma}
  \end{equation}
\noindent Compared to Fig.~\ref{fig:B2DstKDiagram}, in these equations we have
replaced $\accentset{\brobor}{A}\phantom{}_{D^{*}}$ with
$A_{\pi^{0}}\accentset{\brobor}{A}\phantom{}_{D}$ or
$A_{\gamma}\accentset{\brobor}{A}\phantom{}_{D}e^{{i\pi}}$. These amplitudes
are defined as: $\overbar{A}_{D}=A(D^{0}\rightarrow
f(\overbar{D}))=A(\overbar{D}^{0}\rightarrow f(D))$ and
$A_{D}=A(\overbar{D}^{0}\rightarrow f(\overbar{D}))=A(D^{0}\rightarrow f(D))$.
$\delta_D$ is the relative strong phase difference between the $D$ and
$\overbar{D}$ mesons decaying to the same final state. 4 analogous charge
conjugate equations can be written for the $B^{+}$ meson. 

When applying GLW method~\cite{GLWRef}, the $D$ meson is reconstructed in the
$CP$-even final states $D\rightarrow KK$ and $D\rightarrow \pi \pi$, therefore
$\delta_D=0$ and $A_{D}=\overbar{A}_{D}$ can also be factorised out. Time integrated decay rates are proportional to the squared magnitude of the amplitudes, therefore we can construct the following equations for $B^{\pm}$ mesons:
  \begin{equation}
		\Gamma_{CP+}^{\pm}(B^{\pm}\rightarrow (D^{*}\rightarrow
[hh]_D\pi^{0}))\propto |A^{\pi^0}_{Bf_{CP}}|^2 \propto 1 + r_{B}^{2} +
2r_{B}\cos(\delta_{B}\mp \phi_{3}) \label{eq:DecayRateGLWpi0}
  \end{equation}
  \begin{equation}
		\Gamma_{CP-}^{\pm}(B^{\pm}\rightarrow (D^{*}\rightarrow
[hh]_D\gamma))\propto |A^{\gamma}_{Bf_{CP}}|^2 \propto 1 + r_{B}^{2} -
2r_{B}\cos(\delta_{B}\mp \phi_{3}) \label{eq:DecayRateGLWgamma}
  \end{equation}
\noindent The strong phase shift of $\pi$ from the
$D^{*}\rightarrow D\gamma$ decay converts the $CP+$ eigenstate from even to odd
in the context of $B^{\pm}\rightarrow D^{(*)0}K^{\pm}$ decays. This provides a
useful extra constraint in the extraction of $\phi_3$. 

In the ADS method~\cite{ADSRef}, the $D$ meson is reconstructed as
$D^{0}\rightarrow K^{+}\pi^{-}$, $D^{0}\rightarrow K^{-}\pi^{+}$ and their
charge conjugates. The former is doubly Cabbibo suppressed with respect to the
latter, therefore we label these as the suppressed (SUP) and favoured (FAV)
modes. Defining $A_{D}=A(D^{0}\rightarrow
K^{-}\pi^{+})=A(\overbar{D}^{0}\rightarrow K^{+}\pi^{-})$ and $\overbar{A}_{D}=
A(D^{0}\rightarrow K^{+}\pi^{-})=A(\overbar{D}^{0}\rightarrow
K^{-}\pi^{+})=A_{D}r_{D}e^{-i\delta_{D}}$, where $r_{D}$ is the amplitude ratio
of the SUP with respect to the FAV $D$ decay mode and $\delta_{D}$ is their
relative phase, we can construct the following decay rate equations for
$B^{\pm}$ mesons:
  \begin{equation}
		\Gamma_{SUP}^{\pm}(B^{\pm}\rightarrow (D^{*}\rightarrow
[K^{\mp}\pi^{\pm}]_D\pi^{0}))\propto |A^{\pi^0}_{Bf_{SUP}}|^2 \propto r_{B}^{2}
+ r_{D}^{2} + 2r_{B}r_{D}\cos(\delta_{B} - \delta_{D} \mp \phi_{3})
\label{eq:DecayRateADSSUPpi0}
  \end{equation}
  \begin{equation}
		\Gamma_{SUP}^{\pm}(B^{\pm}\rightarrow (D^{*}\rightarrow
[K^{\mp}\pi^{\pm}]_D\gamma))\propto |A^{\gamma}_{Bf_{SUP}}|^2 \propto r_{B}^{2}
+ r_{D}^{2} - 2r_{B}r_{D}\cos(\delta_{B} - \delta_{D} \mp \phi_{3})
\label{eq:DecayRateADSSUPgamma}
  \end{equation}
  \begin{equation}
		\Gamma_{FAV}^{\pm}(B^{\pm}\rightarrow (D^{*}\rightarrow
[K^{\pm}\pi^{\mp}]_D\pi^{0}))\propto |A^{\pi^0}_{Bf_{FAV}}|^2 \propto 1 +
r_{B}^{2}r_{D}^{2} + 2r_{B}r_{D}\cos(\delta_{B} + \delta_{D} \mp \phi_{3})
\label{eq:DecayRateADSFAVpi0}
  \end{equation}
  \begin{equation}
		\Gamma_{FAV}^{\pm}(B^{\pm}\rightarrow (D^{*}\rightarrow
[K^{\pm}\pi^{\mp}]_D\gamma))\propto |A^{\gamma}_{Bf_{FAV}}|^2 \propto 1 +
r_{B}^{2}r_{D}^{2} - 2r_{B}r_{D}\cos(\delta_{B} + \delta_{D} \mp \phi_{3})
\label{eq:DecayRateADSFAVgamma}
  \end{equation}
\\
\noindent The SUP mode is known as the ADS mode. $r_{D}$ and $r_{B}$ are of
similar magnitudes so the ADS mode possesses particularly high sensitivity to
$\phi_3$ as the relative size of the interference term is large. In the usual
ADS analysis~\cite{ADSRef}, two independent equations are obtained with three
unknowns, $r_B$, $\delta_B$ and $\phi_3$ ($r_{D}$ and $\delta_D$ are known from
measurements of $D$ decays). Performing the same method using
$B^{\pm}\rightarrow D^{*}K^{\pm}$ decays results in four independent equations,
Eq.~\eqref{eq:DecayRateADSSUPpi0} and Eq.~\eqref{eq:DecayRateADSSUPgamma}. This
enhanced ADS method therefore has the potential to measure $\phi_3$ without
information from other modes.

\section{\normalsize $CP$ Observables}
We create observables that do not depend on absolute efficiencies by taking
ratios. The physics observables that will be measured in this analysis are:

\begin{enumerate}
	\item The $CP$ asymmetries between $B^{-}$ and $B^{+}$ mesons decaying to $CP$
eigenstates, where $hh=\pi \pi /KK$:
		\begin{equation}
			A_{hh}^{\pi^{0}/\gamma}=\frac{\Gamma(B^{-}\rightarrow
			[[h^{+}h^{-}]_{D}\pi^{0}/\gamma]_{D^{*}}K^{-})-\Gamma(B^{+}\rightarrow
			[[h^{+}h^{-}]_{D}\pi^{0}/\gamma]_{D^{*}}K^{+})}{\Gamma(B^{-}\rightarrow
			[[h^{+}h^{-}]_{D}\pi^{0}/\gamma]_{D^{*}}K^{-})+\Gamma(B^{+}\rightarrow
			[[h^{+}h^{-}]_{D}\pi^{0}/\gamma]_{D^{*}}K^{+})} \label{eq:Aglw}
		\end{equation}
	\item The ratios of $D\rightarrow hh$, where $hh=\pi \pi /KK$, to the favoured mode,
scaled by the branching fraction:
		\begin{equation}
			R_{hh}^{\pi^{0}/\gamma}=\frac{\Gamma(B^{-}\rightarrow
			[[h^{+}h^{-}]_{D}\pi^{0}/\gamma]_{D^{*}}K^{-})+\Gamma(B^{+}\rightarrow
			[[h^{+}h^{-}]_{D}\pi^{0}/\gamma]_{D^{*}}K^{+})}{\Gamma(B^{-}\rightarrow
			[[K^{-}\pi^{+}]_{D}\pi^{0}/\gamma]_{D^{*}}K^{-})+\Gamma(B^{+}\rightarrow
			[[K^{+}\pi^{-}]_{D}\pi^{0}/\gamma]_{D^{*}}K^{+})} \times
			\frac{\mathcal{B}(D^{0}\rightarrow
			K^{-}\pi^{+})}{\mathcal{B}(D^{0}\rightarrow h^{-}h^{+})} \label{eq:Rglw}
		\end{equation}
	\item The ratios of the ADS to the favoured mode, for $B^{\pm}$ decays. We
measure $R_{\pm}$ as opposed to $A_{ADS}$ and $R_{ADS}$ because the latter are
statistically correlated. It should be noted that this constitutes 4 independent equations.
		\begin{equation}
			R_{\pm}^{\pi^{0}/\gamma}=\frac{\Gamma(B^{\pm}\rightarrow
			[[K^{\mp}\pi^{\pm}]_{D}\pi^{0}/\gamma]_{D^{*}}K^{\pm})}{\Gamma(B^{\pm}\rightarrow
			[[K^{\pm}\pi^{\mp}]_{D}\pi^{0}/\gamma]_{D^{*}}K^{\pm})} \label{eq:Rads}
		\end{equation}
\end{enumerate}
\noindent Direct $CP$ violation in $D$ decays is considered negligible, therefore
$A_{KK}^{\pi^{0}}=A_{\pi\pi}^{\pi^{0}}=A_{CP+}$,
$A_{KK}^{\gamma}=A_{\pi\pi}^{\gamma}=A_{CP-}$,
$R_{KK}^{\pi^{0}}=R_{\pi\pi}^{\pi^{0}}=R_{CP+}$ and
$R_{KK}^{\gamma}=R_{\pi\pi}^{\gamma}=R_{CP-}$.

We can express these ratios in terms of the parameters of interest:
  \begin{equation}
    A_{CP\pm}=\frac{\pm 2r_{B}\sin(\delta_{B})\sin(\phi_{3})}{1+r_{B}^{2}\pm 2r_{B}\cos(\delta_{B})\cos(\phi_{3})}
    \label{eq:ACP}
  \end{equation}
  \begin{equation}
    R_{CP\pm}=\frac{1 + r_{B}^{2} \pm 2r_{B}\cos(\delta_{B})\cos(\phi_{3})}{1+r_{B}^{2}r_{D}^{2} \pm 2r_{B}r_{D}\cos(\delta_{B}+\delta_{D})\cos(\phi_{3})}
    \label{eq:RCP}
  \end{equation}
  \begin{equation}
    A_{K\pi}^{\pi^{0}}=\frac{2r_{B}\sin(\delta_{B}+\delta_{D})\sin(\phi_{3})}{1+r_{B}^{2}r_{D}^{2}+2r_{B}r_{D}\cos(\delta_{B}+\delta_{D})\cos(\phi_{3})}
    \label{eq:AfavPi0}
  \end{equation}
  \begin{equation}
    R_{\pm}^{\pi^{0}}=\frac{r_{B}^{2} + r_{D}^{2} +
    2r_{B}r_{D}\cos(\delta_{B}-\delta_{D}\mp
    \phi_{3})}{1+r_{B}^{2}r_{D}^{2}+2r_{B}r_{D}\cos(\delta_{B}+\delta_{D}\mp
    \phi_{3})} \label{eq:RPlusMinusPi0}
  \end{equation}
  \begin{equation}
    A_{K\pi}^{\gamma}=\frac{2r_{B}\sin(\delta_{B}+\delta_{D})\sin(\phi_{3})}{1+r_{B}^{2}r_{D}^{2}-2r_{B}r_{D}\cos(\delta_{B}+\delta_{D})\cos(\phi_{3})}
    \label{eq:AfavGamma}
  \end{equation}
  \begin{equation}
    R_{\pm}^{\gamma}=\frac{r_{B}^{2} + r_{D}^{2} -
    2r_{B}r_{D}\cos(\delta_{B}-\delta_{D}\mp
    \phi_{3})}{1+r_{B}^{2}r_{D}^{2}-2r_{B}r_{D}\cos(\delta_{B}+\delta_{D}\mp
    \phi_{3})} \label{eq:RPlusMinusGamma}
  \end{equation}

For $B^{\pm}\rightarrow D^{*}K^{\pm}$ decays, the current values for $r_B$ and
$\delta_B$ are $0.119^{+0.018}_{-0.019}$ and $(-49^{+12}_{-15})$\degree,
respectively~\cite{website:CKMFitter}. GLW analyses of $B^{\pm}\rightarrow
D^{*}K^{\pm}$ decays have been successfully studied by the
Belle~\cite{BelleGLW}, BaBar~\cite{BaBarGLW} and LHCb~\cite{PartReco}
collaborations. The latter was performed recently using a partially
reconstructed analysis, where the neutral particle was not included in the final
state. No results have yet been published on the ADS mode from the LHCb
collaboration.   
% In 2014, the BaBar experiment attempted the ADS analysis, but in the absence
% of significant signal set the upper limit $r_B<0.21$ ($90\%$ C.L.)
% \cite{BaBarADS}. 

In the following report, $B^{\pm}\rightarrow D^{*}\pi^{\pm}$ decays have also
been studied. These decays are almost kinematically identical to
$B^{\pm}\rightarrow D^{*}K^{\pm}$ decays but have a larger branching fraction.
The higher statistics have been exploited when developing the event selection
and mass fit for this analysis, however the value of $r_B$ for $D^*\pi$ final
states is $\sim 5\%$ that of $D^*K$ final states, therefore decays of this
type are less sensitive to $\phi_3$.

\section{The LHCb Detector and Neutral Particle Reconstruction} \label{detector}
The LHCb~\cite{LHCbDetector} detector is located at the Large Hadron Collider,
a 27km circular $pp$ collider located at CERN. LHCb has been designed to study
$b$ and $c$ quarks, which are predominantly produced in the forward/backward
direction from gluon interactions. For this reason, LHCb is a single-arm
forward spectrometer covering the pseudorapidity region $2 < \eta < 5$, where
$\eta = -\ln (\tan \theta / 2)$. Here, $\theta$ is the angle between the
particle's momentum vector and the beam axis. During Run1 (Run2), the LHC
operated with a collision frequency of 20 MHz (40 MHz). At the interaction
point of LHCb the transverse overlap of the 2 beams is reduced so that on
average 1.7 $pp$ interactions occur per bunch crossing. To reduce the rate of
data uptake, online event selection consists of a hardware and software
trigger. The hardware trigger uses information from the calorimeter and muon
systems, and reduces the rate of information that needs to be stored down to 1
MHz, allowing latency for the full detector to be read out for the software
trigger, which involves a full event reconstruction. Events that pass the
software trigger are read out at a rate of 3 kHz (2011), 5 kHz (2012) and 12.5
kHz (Run2).

The full description of the LHCb detector can be found in \cite{LHCbDetector},
whilst the relevant components will be described here.  The vertex locator, a
semiconductor silicon detector with a transverse resolution of $\mathcal{O}(10$
$\mu$m), sits $8$ mm from the beam line in order to identify the $B$ and $D$
decay vertices crucial for this analysis. The rest of the tracking system is
located further from the $pp$ interaction point and consists of the Tracker
Turicensis (TT), positioned upstream of the dipole magnet, and tracking stations
T1-T3, downstream of the magnet. The TT and the inner tracker (IT) of T1-T3 are
also silicon microstrip detectors, whilst the outer tracker (OT) of the tracking
stations consists of straw drift tube detectors. The dipole magnet itself has an
integrated magnetic field $\int B dl = 4$ Tm.  Positive and negatively charged
particles bend in opposite directions in this field, therefore, to avoid charge
detection asymmetries, it's direction is reversed regularly during data taking.
There are also two Ring Imaging Cherenkov detectors (RICH1 and RICH2), placed at
different distances from the PV to provide particle identification (PID) of
kaons, pions and protons.

The reconstruction of neutrals involves the calorimeter system, consisting of
the electromagnetic (ECAL) and hadronic (HCAL) calorimeters preceded by the
Silicon Pad Detector (SPD) and Preshower (PS), separated by a lead wall. The
ECAL consists of alternating scintillating tiles and lead plates, and the HCAL
iron plates interspaced with scintillating tiles aligned parallel to the beam
pipe. The SPD identifies whether incident particles are charged or neutral,
whilst the PS determines whether they are electrons (if charged) or photons (if
neutral). Both the SPD and PS consist of scintillating pads and all light
produced in the system is sent down wavelength-shifting fibres and transmitted
to photomultiplier tubes.

Once a candidate has been identified as a $\gamma$ or $\pi^0$ from its energy
deposition sequence in the calorimeter system, reconstruction is performed using
the ECAL \cite{NeutralReconstruction}. Energy deposits in ECAL cells are grouped
together into clusters by applying a $3\times3$ cell pattern around the local
energy deposition maxima. If clusters overlap, a convergent iterative procedure
is applied that redistributes the cell energy between the clusters according to
their total energies.  Defining: 
\begin{equation}
  \chi^2_{2D}=(\vec{r}_{tr}-\vec{r}_{cl})^T(C_{tr}+S_{cl})^{-1}(\vec{r}_{tr}-\vec{r}_{cl})
  \label{Chi2}
\end{equation}
$\vec{r}_{tr}$ and $\vec{r}_{cl}$ represent the local coordinates of tracks and
clusters, respectively, at the energy-weighted cluster centres. $C_{tr}$ is the
covariance matrix of the $\vec{r}_{tr}$ and $S_{cl}$ is the cluster energy
spread matrix. Photon candidates are identified to be those without any
associated extrapolated track. This means that they must have a $\chi^2_{2D}$ of
at least 4 for every track in the event. 

The photon energy is then the sum of the total cluster energy in the ECAL and
the reconstructed energy deposit in the PS. Its direction is determined using
the assumed origin and cluster centre of the candidate. It should be noted that,
in this report, only unconverted photons have been considered, i.e. those that
have not pair produced upon reaching the SPD.

Neutral pions are then reconstructed as pairs of well separated photons. These
resolved $\pi^0$s have transverse momentum ($p_T$) in the range $p_{T}<2$ GeV/c.
Above this, the daughter photons cannot be identified as individual clusters due
to finite ECAL granularity. These merged $\pi^0$ candidates are not considered
in this analysis.

A charged particle passing through the whole tracking system has a 96\%
probability of being reconstructed. In contrast, photons and $\pi^0$ mesons have a
reconstruction efficiency of around 10\% and 3\%, respectively. This is what
makes this analysis so challenging.

The data presented in this report was taken at center-of-mass energies of $7$
TeV in 2011, $8$ TeV in 2012 (Run1) and $13$ TeV in 2015, 2016 and 2017 (Run2),
resulting in a total integrated luminosity of $6.6$ $\text{fb}^{-1}$.

\section{Selections} \label{selections}

Signal candidates are initially chosen to have the decay chain and event
topology of the type $B^{\pm}\rightarrow Dh^{\pm}$. A neutral particle is then
combined with the $D$ meson to make a $B^{\pm}\rightarrow D^*h^{\pm}$ candidate.
A kinematic fit is performed to the entire decay chain of each candidate, with
vertex constraints applied to both the $B^{\pm}$ and $D$ daughters, and $D$
(1864.8 MeV/$c^2$) and $\pi^0$ (135.0 MeV/$c^2$) (when present) mesons are
constrained to their known masses.

Candidates must fulfil the hardware trigger, meaning that decay products of the
signal candidate detected by the HCAL, or events containing at least one
candidate from elsewhere in the event, must lie above a fixed threshold in
transverse energy.  The software trigger must also be passed. This places
requirements on the quality of tracks, which are then combined one-by-one, and
identified as having either 2-, 3-, or 4-body topology depending on their
distance of closest approach. 

The $D$ meson is required to be within $\pm$25 MeV/$c^2$ of the known $D^0$
mass, which corresponds to approximately three times the mass resolution.  When
a neutral pion is present, it is required to lie within the asymmetric mass
window of $110 < m[\pi^0] < 185$ MeV/$c^2$. In order to remove combinatorial
backgrounds from low energy photons in the event, transverse momentum
requirements are placed on the neutral particles. In the $D^*\rightarrow
D\gamma$ final state, $\gamma_{P_T}>350$ MeV/$c^2$. In the $D^*\rightarrow
D\pi^0$ mode, $\pi^0_{P_T}>350$ MeV/$c^2$, and the $\pi^0$ secondary photons are
required to have $P_T>200$ MeV/$c^2$.

Particle identification (PID) information is obtained from the RICH detector.
The charged pion and kaon from the $B^{\pm}$ decay vertex are required to occupy
the momentum regions $5 < p < 100$ GeV/$c$ and $0.5 < p_T < 10$ GeV/$c$, to lie
within the RICH's kinematic range. PID cuts are placed on these particles, and
on the $D$ meson daughters, to identify them as pions or kaons. This prevents
the charged meson from the $B^{\pm}$ decay vertex and $D$ meson candidates from
appearing in more than one category. Cross feed between the final states
$D^{0}\rightarrow K^-\pi^+, K^-K^+, \pi^-\pi^+$ is negligible. When both $D$
secondaries are misidentified, however, there is some contamination of
Cabibbo-favoured $B^{\pm}\rightarrow (D^{*0}\rightarrow [K^-\pi^+])h^{\pm}$
decay in the $B^{\pm}\rightarrow (D^{*0}\rightarrow [K^+\pi^-])h^{\pm}$ sample,
due to its relatively large branching fraction. To suppress this cross-feed, a
veto is applied to $D$ final states containing and pion and kaon combination.
The $D^0$ mass is reconstructed with the mass hypothesis of the daughters
swapped, so the kaon is reconstructed whilst assigned the mass of a pion and the
pion is reconstructed whilst assigned the mass of a kaon. $D$ candidates
reconstructed in this way that lie within 15 MeV/$c^2$ of the nominal $D$ mass
are removed, which ensures any cross-feed is negligible.

Combinatorial backgrounds are supressed using two stages of multivariate
analysis (MVA). During the first stage, we try to discriminate against fake $D$
candidates, formed from the combination of random tracks disguised as $D$
daughters, and fake $B$ candidates, made up of a $D$ meson plus a random track.
The use of Boosted Decision Trees (BDTs)~\cite{RegressionTrees} in isolating
$B^{\pm}\rightarrow [h^{\pm}h^{\mp}]_{D}h^{\pm}$ decays has been well
established by the 2-body ADS/GLW analysis of this mode~\cite{B2DKD2hh}. We
therefore employ a similar strategy here. 

\begin{figure}[H]
\centering
\begin{tikzpicture}[scale=0.48]
\node[anchor=south west,inner sep=0] (image) at (0,0)
  {\includegraphics[width=0.48\textwidth]{figures/D0hMvsDeltaMpi0}};
\begin{scope}[x={(image.south east)},y={(image.north west)}]
% \draw[help lines,xstep=.1,ystep=.1] (0,0) grid (1,1);
% \foreach \x in {0,1,...,9} { \node [anchor=north] at (\x/10,0) {0.\x}; }
% \foreach \y in {0,1,...,9} { \node [anchor=east] at (0,\y/10) {0.\y}; }
\draw[green,ultra thick] (0.533,0.0899) rectangle (0.84,0.899);
\node[green,ultra thick] at (0.7,0.5) {BDT1};
\draw[red,ultra thick] (0.2,0.355) rectangle (0.39,0.899);
\node[red,ultra thick] at (0.295,0.62) {BDT2};
\end{scope}
\end{tikzpicture}
\begin{tikzpicture}[scale=0.48]
\node[anchor=south west,inner sep=0] (image) at (0.5,0)
  {\includegraphics[width=0.48\textwidth]{figures/D0hMvsDeltaMgamma}};
\begin{scope}[x={(image.south east)},y={(image.north west)}]
% \draw[help lines,xstep=.1,ystep=.1] (0,0) grid (1,1);
% \foreach \x in {0,1,...,9} { \node [anchor=north] at (\x/10,0) {0.\x}; }
% \foreach \y in {0,1,...,9} { \node [anchor=east] at (0,\y/10) {0.\y}; }
\draw[green,ultra thick] (0.533,0.12) rectangle (0.84,0.899);
\node[green,ultra thick] at (0.7,0.55) {BDT1};
\draw[red,ultra thick] (0.2,0.499) rectangle (0.39,0.899);
\node[red,ultra thick] at (0.295,0.7) {BDT2};
\end{scope}
\end{tikzpicture}
\caption{\small{$\Delta_M=m[D^{*}] - m[D]- (m[\pi^0] +
m[\pi^0]_{PDG})$, where the part in brackets refers to the definition for the
$D^*\rightarrow D\pi^0$ decay mode, is plotted against partially reconstructed
$m[Dh]$ for both $D^*$ meson decay modes. The training regions in data for
both stages of MVA are shown by the labelled boxes.}}
\label{fig:trainingData}
\end{figure}

Data side-bands for the high statistics modes $B^{\pm}\rightarrow(D^{*}
\rightarrow [K^{\pm}\pi^{\mp}]_{D} \pi^0 /\gamma)\pi^{\pm}$ are provided as a
background sample for the BDT. Here, side-bands constitutes data that falls in
the range 5800 $ < m[D\pi] <$ 6800 MeV/$c^2$, as shown in
Fig.~\ref{fig:trainingData}. The signal sample consists of simulated events of
the same decay mode. The training variables used are related to the decay
kinematics and topology of tracks, and are displayed in
Table~\ref{table:bdt1TrainingVar}. 

The working point of BDT1 $>$ 0.05 was chosen by performing fits to the $D$ mass
distribution of the high statistics mode, in BDT steps of 0.05. The extracted
signal yield was then scaled to the ADS mode using the branching ratio
$\mathcal{BR} = \frac{\mathcal{BF}(D^0 \rightarrow K^- \pi^+)}{\mathcal{BF}(D^0
\rightarrow K^+ \pi^-)}$. The optimisation was performed using the
significanc:
\begin{equation}
S = \frac{Sig. Yield}{\sqrt{Sig. Yield \times Bkg. Yield}}
\label{D_branching_ratio}
\end{equation}

BDTs were also employed in the second stage MVA, designed to select the correct
neutral and therefore the true $D^*$ candidate. Separate BDTs were trained for
the two $D^*$ final states, $D\pi^0$ and $D\gamma$. The signal sample provided
was made up of the same simulation samples as were used in the first stage. The
background samples consist of data in the $\Delta_M=m[D^{*}] - m[D]- (m[\pi^0]
+ m[\pi^0]_{PDG})$ upper side-bands, 250 $ < \Delta_{M} < $ 500 MeV/$c^2$, but
$m[Dh]$ signal region, as shown in Fig.~\ref{fig:trainingData}. The training
variables are a mixture of momentum variables and parameters quantifying photon
quality, as displayed in Table.~\ref{table:bdt2TrainingVar}. 



\section{2D Mass Fit} \label{massfit}
\section{Conclusions and Ongoing Work} \label{conclusion}

\printbibliography[heading=bibintoc,{title=References}]

\end{document}
